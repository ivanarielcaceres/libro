% !TeX program = PdfLaTeX
% !TeX encoding = ISO-8859-1
% !TeX spellcheck = es_ES, en_US




\documentclass[a4paper,12pt]{book}
  
	\usepackage{ae}
	\usepackage[T1]{fontenc}
	\usepackage{geometry}
	\usepackage{graphicx}  
	\usepackage{fancyhdr} 
	
	%Usar cuando queremos ver el area de impresion(muy bueno)
	%\usepackage{showframe}
	
	
	%Configura el footer con numeros de pagina CE=CENTER EVENT, CO=CENTER ODD
	\pagestyle{fancy}	
	\fancyhf{}
	\fancyfoot[CE,CO]{\ifnum\value{page}<4\relax\else\thepage\fi}
	%elimina el subrayado del header
	\renewcommand{\headrulewidth}{0pt}
	%muestra un subrayado despues del footer
	%\renewcommand{\footrulewidth}{0.4pt}
	


  
	\begin{document} 	

		%Cubiertas o tapas del libro
		
\begin{center}
	\includegraphics[width=30mm]{uca.png}
\end{center}	

\begin{center}
	\fontsize{16}{16}\selectfont
	UNIVERSIDAD CATOLICA\\
	``NUESTRA SE�ORA DE LA ASUNCI�N"\\
	CAMPUS ALTO PARAN�\\
	FACULTAD DE CIENCIAS Y TECNOLOG�A
\end{center}


\vspace*{2\baselineskip}
\input{globales/titulo}

\vspace*{2\baselineskip}
\begin{center}
	\fontsize{12}{12}\selectfont
	Proyecto Final de Graduaci�n presentado a la Facultad de Ciencias y
	Tecnolog�a como requisito obligatorio para la obtenci�n del t�tulo de
	Lic. en An�lisis de Sistemas
\end{center}

%3 saltos de linea
\vspace*{3\baselineskip}
\begin{center}	
	\textbf{
	\input{globales/autor1}
	\\
	\input{globales/autor2}
	}
\end{center}


%llena el espacio vacio
\vfill
\begin{center}
	Hernandar�as, diciembre de 2015
\end{center}



		\begin{center}
	\fontsize{16}{16}\selectfont
	UNIVERSIDAD CATOLICA\\
	``NUESTRA SE�ORA DE LA ASUNCI�N"\\
	CAMPUS ALTO PARAN�\\
	FACULTAD DE CIENCIAS Y TECNOLOG�A
\end{center}
\vspace*{5\baselineskip}
\input{globales/titulo}
\vspace*{5\baselineskip}
\begin{center}	
	\textbf{
	\input{globales/autor1}
	\\
	\input{globales/autor2}
	}
\end{center}
\vfill
\begin{center}
	\fontsize{11}{11}\selectfont
	Ricardo Luis Brunelli Montero, Ing.\\
	Asesor
\end{center}
		\begin{center}	
	\textbf{
	\input{globales/autor1}
	\\
	\input{globales/autor2}
	}
\end{center}

\vspace*{10\baselineskip}
\input{globales/titulo}

\vfill
\hspace{.25\textwidth} % posicionando a minipage
\begin{minipage}{.8\textwidth}
	Proyecto de Fin de Carrera presentado como requisito parcial para optar al t�tulo de Lic. en An�lisis de Sistemas.
	
\end{minipage}


\hspace{.25\textwidth} % posicionando a minipage
\begin{minipage}{.8\textwidth}
	Facultad de Ciencias y Tecnolog�a, Universidad Cat�lica ``Nuestra Se�ora de la Asunci�n"\\
	
	Tutor: Ing. Ricardo Luis Brunelli Montero		
\end{minipage}



%llena el espacio vacio
\vfill
\begin{center}
	Hernandar�as, abril de 2016
\end{center}



		\vspace*{\fill}
\hfill\noindent\fbox{%
	\parbox{11cm}{\raggedright
	Landaida Duarte, Ariel Hern�n;C�ceres Ca�ete, Iv�n Ariel. (2016); APP-NAME, una aplicacion para ayudar a las personas a analizar, visualizar y compartir informaci�n r�pidamente. Hernandarias, Universidad Catolica. 110 p.\\
	
	\textbf{Tutor:} Ing.. Ricardo Luis Brunelli Montero.\\
	\textbf{Defensa de Proyecto de Fin de Carrera.}\\
	\textbf{Palabras clave:} Data Discovery, Business Intelligence.		
	}%
}
				
		
		\title{Data Discovery Paraguay}
		\author{Ariel Hern�n Landaida Duarte, Iv�n Ariel Caceres Ca�ete}
		%oculta la fecha por defecto
		\date{}
		
		%\maketitle		
		%Data Discovery	  

	  
		%\clearpage\null
		%Oculta la palabra Chapter y muestra solo el nombre que define para cada capitulo
		\renewcommand{\chaptername}{CAP�TULO}
		
		%Insertar capitulos
		\chapter{\centerline{Introducci�n}}
%\chapter{Introducci�n}
%\chaptermark{version for header}
\markboth{Introducci�n}{}
Es una arquitectura de BI dirigido a informes interactivos y datos explorables de m�ltiples fuentes. De acuerdo con Gartner,  la firma de investigaci�n y asesoramiento de la tecnolog�a de la informaci�n en Am�rica, data discovery se ha convertido en una arquitectura dominante en 2012.

Jill Dyche llama data discovery ?descubrimiento del conocimiento? y la define como "la detecci�n de patrones en los datos?. Estos patrones son muy espec�ficos y aparentemente arbitrarios para especificar, y el analista estar�a jugando un juego de adivinanzas tratando de descifrar todos los posibles patrones en la base de datos. En cambio, las herramientas especiales de software para data discovery encuentra los patrones y dicen al analista lo que y donde encontrarlos.

El actual vicepresidente(2013-2014) de buenas pr�cticas en SAS Institute dice que no es de extra�ar que la definici�n de data discovery se asemeja a la definici�n de miner�a de datos:

"La miner�a de datos un subcampo interdisciplinario de ciencias de la computaci�n, es el proceso computacional de descubrir patrones en grandes conjuntos de datos involucrando m�todos de la  inteligencia artificial, aprendizaje autom�tico, estad�stica y sistemas de bases de datos. El objetivo general del proceso de miner�a de datos es extraer informaci�n de un conjunto de datos y transformarla en una estructura comprensible para su uso posterior. Aparte de la etapa de an�lisis en bruto, trata de bases de datos y gesti�n de datos, preprocesamiento de datos, consideraciones de modelo e inferencia, m�tricas interesantes, consideraciones de complejidad, postprocesado de las estructuras descubiertas, visualizaci�n y actualizaci�n en l�nea ".

Tambi�n puede ser referido como Business Discovery.
\section{Objetivos}
tdfasdfasdfasdf	
	  
		\section[dlsadk]{Business Intelligence}
		Business Intelligence (from reference \cite{kumari2013business})
		
		
		\renewcommand{\bibname}{Bibliograf�a}
		\bibliographystyle{plain}
		\bibliography{contenido/bibFile}
	

	\end{document}
  
  
  
  
  
  
  