% !TeX program = PdfLaTeX
% !TeX encoding = ISO-8859-1
% !TeX spellcheck = es_ES, en_US




\documentclass[a4paper,12pt]{book}
  
	\usepackage{ae}
	\usepackage[T1]{fontenc}
	\usepackage{geometry}
	\usepackage{graphicx} 
	\usepackage{wrapfig}
	\graphicspath{ {figuras/} } 
	\usepackage{fancyhdr} 
	\usepackage{sectsty}
	\usepackage{tocloft}
	\usepackage[nopostdot,style=super,nonumberlist,toc]{glossaries}
	\usepackage{float}
	\restylefloat{table}

	
	
	%Centra el titulo de los capitulos
	\chapterfont{\centering}
	%Centra el titulo de las secciones
	%\sectionfont{\centering}
	%Centra el titulo de las sub-secciones
	%\subsectionfont{\centering}
	
	

	
	%Usar cuando queremos ver el area de impresion(muy bueno)
	%\usepackage{showframe}
	
	
	%Configura el footer con numeros de pagina CE=CENTER EVENT, CO=CENTER ODD
	%%\pagestyle{fancy}	
	%%\fancyhf{}
	%%\fancyfoot[CE,CO]{\ifnum\value{page}<17\relax\else\thepage\fi}
	%elimina el subrayado del header
	%%\renewcommand{\headrulewidth}{0pt}
	%muestra un subrayado despues del footer
	%\renewcommand{\footrulewidth}{0.4pt}
	
	
	% Page style for preliminary pages.
	\fancypagestyle{preliminary}{
		\fancyhf{}% Clear header/footer
		\fancyfoot[C]{\thepage}% Footer
		\renewcommand{\headrulewidth}{0pt}% No header rule
	}
	
	% Page style for main matter.
	\fancypagestyle{mainmatter}{
		\fancyhf{}% Clear header/footer
		\fancyfoot[C]{\thepage}% Footer
		\renewcommand{\headrulewidth}{0pt}% No header rule
	}
	
	
	% Glosario
	%\usepackage[doublespacing]{setspace}
	
	\setlength{\glsdescwidth}{0.8\linewidth}
	\newglossarystyle{mylong}{
		\glossarystyle{long}
		\renewcommand*{\glossaryentryfield}[5]{%
			\glsentryitem{##1}\glstarget{##1}{##2} & ............................................................. ##3\glspostdescription\space ##5\\[8pt]}%
		\renewcommand{\glsgroupskip}{}
	}
	\makeglossaries% Inicia

	
	\begin{document}
		\pagestyle{preliminary}\pagenumbering{roman}
		\fancypagestyle{plain}{\pagestyle{preliminary}}% Correct plain page style
		
		%Cubiertas o tapas del libro
		
\begin{center}
	\includegraphics[width=30mm]{uca.png}
\end{center}	

\begin{center}
	\fontsize{16}{16}\selectfont
	UNIVERSIDAD CATOLICA\\
	``NUESTRA SE�ORA DE LA ASUNCI�N"\\
	CAMPUS ALTO PARAN�\\
	FACULTAD DE CIENCIAS Y TECNOLOG�A
\end{center}


\vspace*{2\baselineskip}
\input{globales/titulo}

\vspace*{2\baselineskip}
\begin{center}
	\fontsize{12}{12}\selectfont
	Proyecto Final de Graduaci�n presentado a la Facultad de Ciencias y
	Tecnolog�a como requisito obligatorio para la obtenci�n del t�tulo de
	Lic. en An�lisis de Sistemas
\end{center}

%3 saltos de linea
\vspace*{3\baselineskip}
\begin{center}	
	\textbf{
	\input{globales/autor1}
	\\
	\input{globales/autor2}
	}
\end{center}


%llena el espacio vacio
\vfill
\begin{center}
	Hernandar�as, diciembre de 2015
\end{center}



		\clearpage\null
		\begin{center}
	\fontsize{16}{16}\selectfont
	UNIVERSIDAD CATOLICA\\
	``NUESTRA SE�ORA DE LA ASUNCI�N"\\
	CAMPUS ALTO PARAN�\\
	FACULTAD DE CIENCIAS Y TECNOLOG�A
\end{center}
\vspace*{5\baselineskip}
\input{globales/titulo}
\vspace*{5\baselineskip}
\begin{center}	
	\textbf{
	\input{globales/autor1}
	\\
	\input{globales/autor2}
	}
\end{center}
\vfill
\begin{center}
	\fontsize{11}{11}\selectfont
	Ricardo Luis Brunelli Montero, Ing.\\
	Asesor
\end{center}
		\clearpage\null
		\begin{center}	
	\textbf{
	\input{globales/autor1}
	\\
	\input{globales/autor2}
	}
\end{center}

\vspace*{10\baselineskip}
\input{globales/titulo}

\vfill
\hspace{.25\textwidth} % posicionando a minipage
\begin{minipage}{.8\textwidth}
	Proyecto de Fin de Carrera presentado como requisito parcial para optar al t�tulo de Lic. en An�lisis de Sistemas.
	
\end{minipage}


\hspace{.25\textwidth} % posicionando a minipage
\begin{minipage}{.8\textwidth}
	Facultad de Ciencias y Tecnolog�a, Universidad Cat�lica ``Nuestra Se�ora de la Asunci�n"\\
	
	Tutor: Ing. Ricardo Luis Brunelli Montero		
\end{minipage}



%llena el espacio vacio
\vfill
\begin{center}
	Hernandar�as, abril de 2016
\end{center}



		\clearpage\null
		\vspace*{\fill}
\hfill\noindent\fbox{%
	\parbox{11cm}{\raggedright
	Landaida Duarte, Ariel Hern�n;C�ceres Ca�ete, Iv�n Ariel. (2016); APP-NAME, una aplicacion para ayudar a las personas a analizar, visualizar y compartir informaci�n r�pidamente. Hernandarias, Universidad Catolica. 110 p.\\
	
	\textbf{Tutor:} Ing.. Ricardo Luis Brunelli Montero.\\
	\textbf{Defensa de Proyecto de Fin de Carrera.}\\
	\textbf{Palabras clave:} Data Discovery, Business Intelligence.		
	}%
}
		\begin{center}	
	\textbf{
	\input{globales/autor1}
	\\
	\input{globales/autor2}
	}
\end{center}

\vspace*{5\baselineskip}
\input{globales/titulo}


\vspace*{5\baselineskip}
\hspace{.1\textwidth} % posicionando a minipage
\begin{minipage}{.8\textwidth}
	Proyecto de Fin de Carrera presentado como requisito parcial para optar al t�tulo de Lic. en An�lisis de Sistemas.
	
\end{minipage}

\vspace*{2\baselineskip}
\begin{center}
	Mesa Examinadora\\	
	
	\vspace*{1\baselineskip}
	\makebox[3in]{\hrulefill}
	\par\noindent
	Prof. Nelida Elizabeth Delgado, Lic.\\
	Presidente de Mesa\\
	\vspace*{1\baselineskip}
	\makebox[3in]{\hrulefill}
	\par\noindent
	Prof. Manuel Chamorro Alderete, Ing.\\	
	Miembro de Mesa\\
	\vspace*{1\baselineskip}
	\makebox[3in]{\hrulefill}
	\par\noindent
	Prof. Ricardo Luis Brunelli, Ing.\\
	Presidente de Mesa\\
\end{center}
\vspace*{4\baselineskip}
Nota obtenida:

\vspace*{1\baselineskip}
\hspace{.5\textwidth} % posicionando a minipage
Hernandarias, 25 de Diciembre de 2015


		\include{cubierta/dedicatoria}
		\begin{center}
		\fontsize{16}{16}\selectfont
	\textbf{Agradecimientos}
\end{center}


\vfill
\begin{center}
	A Dios por la fortaleza que siempre nos ha dado en todos los momentos de nuestra vida.
	
	Gracias, a nuestro tutor, el Ing. Ricardo Luis Brunelli. Gracias por su paciencia, dedicaci�n, motivaci�n, criterio y aliento. Ha hecho f�cil lo dif�cil. Ha sido un privilegio poder contar con su gu�a y ayuda.
	
	A nuestra familia quienes por ellos somos lo que somos. A nuestros padres por su apoyo, consejos, comprensi�n, amor, ayuda en los momentos dif�ciles, y por ayudarnos con los recursos necesarios para estudiar. Nos han dado todo lo que somos como persona, valores, principios, car�cter, empe�o, perseverancia, y coraje para conseguir nuestros objetivos.
\end{center}
		\begin{center}
	RESUMEN
\end{center}

\vspace*{2\baselineskip}
\chapter*{Abstract}
	Se realiz� una investigaci�n en el �rea de la Inteligencia de Negocios (BI) con datos de la instituci�n p�blica ANDE con la finalidad de estimar un potencial crecimiento de su estructura de acuerdo al crecimiento de la poblaci�n pudiendo as� tener un mejor entendimiento de los datos y facilitar la toma de decisiones.


\vspace*{1\baselineskip}
\textbf{Palabras clave:} Data Discovery, Business Intelligence, Datos de la ANDE.
		\chapter*{\centering ABSTRACT}
	An investigation was conducted in the area of Business Intelligence (BI) with data from the public institution ANDE in order to estimate potential growth of its structure according to the growth of the population and can have a better understanding of the data and facilitate decision-making.


\vspace*{1\baselineskip}
\textbf{Keywords:} Data Discovery, Business Intelligence, Datos de la ANDE.
		%\newacronym{sig:BI}{BI}{\textit{Business Intelligence}}
\newacronym{sig:ANDE}{ANDE}{\textit{Administraci�n Nacional de Electricidad}}
\newacronym{sig:DGEEC}{DGEEC}{\textit{Direcci�n General de Estad�sticas, Encuestas y Censo}}
\newacronym{sig:CPU}{CPU}{\textit{Unidad Central de Procesos}}
\newacronym{sig:ERP}{ERP}{\textit{Sistemas de Planificaci�n de Recursos Empresariales}}
\newacronym{sig:CRM}{CRM}{\textit{Administraci�n basada en la relaci�n con los clientes}}
\newacronym{sig:OLAP}{OLAP}{\textit{Procesamiento nal�tico en l�nea}}

	
		
		
		%Lista de Figuras		
		\newlength{\mylen}		
		\renewcommand{\listfigurename}{�NDICE DE FIGURAS}
		\renewcommand{\figurename}{Figura}
		\renewcommand{\cftfigpresnum}{Figura\enspace}
		\renewcommand{\cftfigaftersnum}{:}
		\settowidth{\mylen}{\cftfigpresnum\cftfigaftersnum}
		\addtolength{\cftfignumwidth}{\mylen}
		\listoffigures

		
		%Lista de Tablas
		\clearpage		
		\renewcommand{\listtablename}{LISTA DE TABLAS}
		\listoftables


		
		%GLOSARIO		
		\clearpage		
		\printglossary[style=mylong,title=Lista de Siglas y Acr�nimos]

		
		%Lista de contenido
		\clearpage				
		\renewcommand{\contentsname}{�NDICE}
		\tableofcontents
		
		\title{Data Discovery Paraguay}
		\author{Ariel Hern�n Landaida Duarte, Iv�n Ariel Caceres Ca�ete}
		%oculta la fecha por defecto
		\date{}
		
		%\maketitle		
		%Data Discovery	  

		
		\mainmatter
		\pagestyle{mainmatter}\pagenumbering{arabic}
		\fancypagestyle{plain}{\pagestyle{mainmatter}}% Correct plain page style
	  
		%\clearpage\null
		%Oculta la palabra Chapter y muestra solo el nombre que define para cada capitulo
		\renewcommand{\chaptername}{CAP�TULO}

		%Insertar capitulos
		\chapter{Introducci�n}
Esta tesis se basa en un trabajo de investigaci�n sobre el estado del arte de lo que denominamos BI (Business intelligence) y data discovery. El objetivo es aplicar los conocimientos adquiridos a nuestro entorno, en este caso aplicados a una instituci�n del estado (la ANDE), para el apoyo a la hora de tomar decisiones.

Este trabajo presenta los siguientes cap�tulos:
El cap�tulo I presenta el planteamiento del problema, los objetivos generales y espec�ficos, se hablar� sobre los conceptos, componentes, tecnolog�as de BI, el framework de Gartner, la cual fue preparado por una de las consultoras l�deres en BI.
El cap�tulo II presenta los diferentes dashboards realizados con la herramienta seleccionada para aplicar data discovery sobre los conjuntos de datos existentes. Se explicar� el cuadrante m�gico de Gartner, el cual ayud� a la selecci�n de la herramienta mencionada. Se demostrar�n ejemplos de an�lisis de datos y la informaci�n capaz de obtener si se hace un buen uso de la herramientas y los datos subyacentes.
\section{Planteamiento del problema}
Actualmente la mayor parte de las instituciones estatales e incluso del sector privado no invierten lo suficiente en tecnolog�a. La mayor�a de las empresas contratan personas que tomen decisiones correctas en la situaci�n que se presente. Estas personas no pueden ejecutar ninguna acci�n correcta si es que no poseen la experiencia en el negocio o  los datos correctos para analizar. La experiencia resulta en organizaciones peque�as, donde con un tiempo se adquiere conocimiento sin necesidad de un apoyo digital, por ejemplo, los productos m�s vendidos o m�s rentables de un negocio. Esto se complica a medida que la organizaci�n crece. Cuando la cantidad de sucursales y variedad de productos es amplio ya se pierde el control de manejarlo sin ayuda de la tecnolog�a. 

La calidad del conocimiento se basa principalmente en que  la informaci�n sea de calidad. Estas informaciones son obtenidas a trav�s de un an�lisis profundo de los datos, por consiguiente estos datos tambi�n deben ser de buena calidad. Si una organizaci�n posee la capacidad de obtener lo que desea por medio de los datos, es seguro que su crecimiento ser� positivo, debido a que tomar� mejores decisiones, y estos ayudar�an a su evoluci�n a lo largo del tiempo.

Existen ocasiones en que la organizaci�n posee los datos suficientes pero no consigue analizarlos o procesarlos por falta conocimiento ya sea t�cnico o de negocio.
		\chapter{Objetivos}

\section{Objetivo General}
En este trabajo aplicamos los conocimientos adquiridos en la investigaci�n a una instituci�n del estado, para demostrar que con los datos correctos podemos detectar situaciones, predecir acontecimientos de inter�s para la instituci�n. Se cruzar�n conjunto de datos de la ANDE y DGEEC.
\section{Objetivos Espec�ficos}
\begin{itemize}

	\item Analizar y cruzamiento de datos para la detecci�n de situaciones de inter�s.
	\item Estudio del estado del arte de BI y Data discovery.
	\item Selecci�n de la herramienta adecuada para este trabajo.
	\item Elaboraci�n de dashboards que reflejen el an�lisis de los datos con la herramienta seleccionada.
	\item Elaboraci�n de gr�ficos de tendencias, proyecciones de consumo y clientes de la instituci�n. 
\end{itemize}


		\section{Justificaci�n}
La principal ventaja de conocer la importancia de los datos,  las capacidades y conocimientos que deben poseer las personas que lo analizan, las que toman las decisiones dentro de la organizaci�n es que podr�n evolucionar y mejorar sus estrategias de negocios, haci�ndolo cada vez m�s eficiente, y con ayuda de esto, evolucionar como organizaci�n misma. Las personas que analizan los datos, obtienen informaci�n dependiendo de la limitaci�n que poseen, es decir, cuando estas personas est�n al tanto de lo que se puede lograr con un an�lisis profundo, las mismas ser�n capaces de asimilarlos a su favor.
		\chapter{\centerline{Introducci�n}}
%\chapter{Introducci�n}
%\chaptermark{version for header}
\markboth{Introducci�n}{}
Es una arquitectura de BI dirigido a informes interactivos y datos explorables de m�ltiples fuentes. De acuerdo con Gartner,  la firma de investigaci�n y asesoramiento de la tecnolog�a de la informaci�n en Am�rica, data discovery se ha convertido en una arquitectura dominante en 2012.

Jill Dyche llama data discovery ?descubrimiento del conocimiento? y la define como "la detecci�n de patrones en los datos?. Estos patrones son muy espec�ficos y aparentemente arbitrarios para especificar, y el analista estar�a jugando un juego de adivinanzas tratando de descifrar todos los posibles patrones en la base de datos. En cambio, las herramientas especiales de software para data discovery encuentra los patrones y dicen al analista lo que y donde encontrarlos.

El actual vicepresidente(2013-2014) de buenas pr�cticas en SAS Institute dice que no es de extra�ar que la definici�n de data discovery se asemeja a la definici�n de miner�a de datos:

"La miner�a de datos un subcampo interdisciplinario de ciencias de la computaci�n, es el proceso computacional de descubrir patrones en grandes conjuntos de datos involucrando m�todos de la  inteligencia artificial, aprendizaje autom�tico, estad�stica y sistemas de bases de datos. El objetivo general del proceso de miner�a de datos es extraer informaci�n de un conjunto de datos y transformarla en una estructura comprensible para su uso posterior. Aparte de la etapa de an�lisis en bruto, trata de bases de datos y gesti�n de datos, preprocesamiento de datos, consideraciones de modelo e inferencia, m�tricas interesantes, consideraciones de complejidad, postprocesado de las estructuras descubiertas, visualizaci�n y actualizaci�n en l�nea ".

Tambi�n puede ser referido como Business Discovery.
\section{Objetivos}
tdfasdfasdfasdf		
		\fontsize{12}{12}\selectfont
\chapter{\centerline{Conceptos}}
		
\section{Aplicaci�n de Data Discovery a datos de instituciones del Estado}

\begingroup
\setlength{\intextsep}{-20pt}%
\setlength{\columnsep}{5pt}%
\begin{wrapfigure}{r}{0.25\textwidth}
	\centering
	\includegraphics[width=\linewidth]{figuras/makeDecision}
	\vspace{-10pt}
	\caption{}
	\label{fig:makeDecision}
\end{wrapfigure}
En Paraguay, e inclusive mundialmente, la mayor�a de las empresas no logran comprender 100\% los datos que generan. La consecuencia de no comprender los datos puede conllevar a tomar una decisi�n equivocada y esa decisi�n puede terminar en algo irreversible y de mucho impacto negativo para la organizaci�n a lo largo del tiempo. La informaci�n es considerada como uno de los recursos m�s importantes en una empresa, porque en base a esto se obtiene el conocimiento.

Aplicando la t�cnica de data discovery podremos detectar irregularidades, predecir acontecimientos futuros y preverlos a tiempo. En este trabajo analizaremos los datos de la ANDE y de la DGEEC, realizando cruzamiento entre ambos, con el objetivo de obtener informaci�n de inter�s para la entidad.
\endgroup 

\subsubsection{Datos de la ANDE y de la DGEEC}

Se cuenta con datos de consumo de energ�a el�ctrica, importe facturado, grupo de consumo (residencial, industrial, exportaci�n, etc), por a�o (2000-2014), departamento y distrito. Estos datos fueron solicitados formalmente a la entidad a trav�s de la Facultad de Ciencias y tecnolog�a de la Universidad Cat�lica, la cual tuvimos una respuesta favorable para proceder.

\subsubsection{Dashboard de control / monitoramiento}
	
En esta secci�n se mostrar�n 4 ejemplos de paneles hechos con los datos de ambas instituciones (DGEEC y ANDE).  Una de las t�cnicas de comparaci�n a ser utilizadas en los gr�ficos es ,  ``Medici�n de tasas de crecimientos``, la cual se calcula el porcentaje de crecimiento que hubo por cada a�o. Por ejemplo, si al cerrar el a�o 2014, la cantidad de clientes lleg� a 1.000.000 y en el a�o 2015 aument� 100.000, esto quiere decir que en el a�o 2015, la tasa de crecimiento de los clientes fue del 10\%, es decir, hubo un crecimiento positivo y la cantidad de clientes ha aumentado el respecto al a�o anterior. Suponiendo que el a�o 2016 la ANDE cierra con un total de 1.000.000 clientes,, su crecimiento ser� de 10\% negativo.
La f�rmula empleada es la que sigue, donde ``n`` es el a�o actual y ``n-1`` el a�o anterior, PIB es una variable que indica, en el caso  de nuestra comparaci�n, la cantidad de clientes que posee la ANDE .
	
\textsc{\begin{figure}[H]
	\centering
	\includegraphics[width=0.7\linewidth]{figuras/formula}
	\caption{Form�la para hallar tasa de crecimiento.}
	\label{fig:formula}
\end{figure}}
	
\section{Dashboard - Clientes Facturados vs Crecimiento Poblacional}
Este dashboard se realiz� a fin de cruzar los datos del DGEEC y la ANDE, se utilizan datos hist�ricos de la poblaci�n y datos de clientes, consumo e importe de la ANDE.

\textsc{\begin{figure}[H]
	\centering
	\includegraphics[width=\linewidth]{figuras/ClientesFacturadosVsCrecimientoPoblacional3}
	\caption{Clientes Facturados vs Crecimiento Poblacional}
	\label{fig:ClientesFacturadosVsCrecimientoPoblacional3}
\end{figure}}

El siguiente gr�fico representa el porcentaje del crecimiento anual de los clientes en comparaci�n con el de la poblaci�n. Podemos observar que la l�nea que pertenece a la tasa de crecimiento de la poblaci�n (azul), fue bajando al pasar los a�os. Esto no quiere decir que la poblaci�n fue disminuyendo, sino que cada a�o el porcentaje de aumento es menor.

\textsc{\begin{figure}[H]
	\centering
	\includegraphics[width=\linewidth]{figuras/ClientesFacturadosVsCrecimientoPoblacional4}
	\caption{Clientes Facturados vs Crecimiento Poblacional}
	\label{fig:ClientesFacturadosVsCrecimientoPoblacional4}
\end{figure}}


\textsc{\begin{figure}[H]
	\centering
	\includegraphics[width=\linewidth]{figuras/ClientesFacturadosVsCrecimientoPoblacional}
	\caption{Clientes Facturados vs Crecimiento Poblacional}
	\label{fig:ClientesFacturadosVsCrecimientoPoblacional}
\end{figure}}

La l�nea amarilla representa al porcentaje del crecimiento de los clientes de la ANDE. Como podemos ver, hay      a�os en que el aumento es muy notorio (2004,2006,2008) y hay a�os en que este es m�nimo(2002,2005,2007). Las l�neas discontinuas representan las tendencias de ambos puntos. Por ejemplo, la cantidad de clientes en el a�o 2001 fue de 959.580, la cual aument� el 4.1\% respecto al a�o anterior. 

\textsc{\begin{figure}[H]
	\centering
	\includegraphics[width=\linewidth]{figuras/ClientesFacturadosVsCrecimientoPoblacional2}
	\caption{Clientes Facturados vs Crecimiento Poblacional}
	\label{fig:ClientesFacturadosVsCrecimientoPoblacional2}
\end{figure}}

En el a�o 2002 la cantidad de clientes ascendi� a 964.449 con un aumento de 4.869, que corresponde a un incremento     del 0.5\% respecto  al a�o 2001. Sin embargo, en el a�o 2003 el incremento fue de 1.\%, la cual representa a un aumento de m�s que el doble del a�o anterior.


\textsc{\begin{figure}[H]
	\centering
	\includegraphics[width=\linewidth]{figuras/ClientesFacturadosVsCrecimientoPoblacional5}
	\caption{Clientes Facturados vs Crecimiento Poblacional}
	\label{fig:ClientesFacturadosVsCrecimientoPoblacional5}
\end{figure}}

\textsc{\begin{figure}[H]
	\centering
	\includegraphics[width=\linewidth]{figuras/ClientesFacturadosVsCrecimientoPoblacional6}
	\caption{Clientes Facturados vs Crecimiento Poblacional}
	\label{fig:ClientesFacturadosVsCrecimientoPoblacional6}
\end{figure}}

En el siguiente gr�fico se realiza una proyecci�n o forecasting donde se muestra que probablemente, si la entidad conserva la misma cantidad de clientes o estos crecen m�nimamente, de igual manera el consumo podr�a aumentar o disminuir dr�sticamente. El aumento dr�stico del consumo, podr�a deberse al aumento de productos electr�nicos que consumen mucha m�s energ�a el�ctrica y tambi�n a que el poder adquisitivo de cada ciudadano ha aumentado. Este comportamiento se observa en la zona de c�lculo de proyecci�n (rojo, azul y amarillo suavizado).



\textsc{\begin{figure}[H]
	\centering
	\includegraphics[width=\linewidth]{figuras/EnergiaConsumidaVsImporteFacturado}
	\caption{Energ�a Consumida vs Importe Facturado}
	\label{fig:EnergiaConsumidaVsImporteFacturado}
\end{figure}}

En el tercer y �ltimo gr�fico de este panel, se muestra la porcentaje del crecimiento anual de los importes facturados y consumo de energ�a. Se puede observar que la facturaci�n de la ANDE acompa�a al consumo de energ�a el�ctrica, exceptuando el a�o 2011, en la cual el importe aument� mas de lo que aument� el consumo de energ�a, sin embargo en el a�o 2013 el importe volvi� a aumentar menos que antes.

\textsc{\begin{figure}[H]
	\centering
	\includegraphics[width=\linewidth]{figuras/EnergiaConsumidaVsImporteFacturado2}
	\caption{Energ�a Consumida vs Importe Facturado}
	\label{fig:EnergiaConsumidaVsImporteFacturado2}
\end{figure}}
\subsection{Dashboard - Estad�stica de consumo de electricidad por sector}
En el primer tab, ``Consumo``, tenemos una estad�stica de consumo por grupo de consumidores de energ�a. Al abrir este panel, observamos que la el grupo de consumidores residencial es la que m�s demanda energ�a, hist�ricamente. Otro dato interesante es la exportaci�n de energ�a el�ctrica, que seg�n el gr�fico, cada vez fue disminuyendo m�s. 
A nivel nacional la zona residencial se ubica en primer lugar con un 39,87\%.  Luego viene el sector industrial, que consume el 22.11\% de la energ�a. El sector comercial es due�a del 17,39\%.


\textsc{\begin{figure}[H]
	\centering
	\includegraphics[width=\linewidth]{figuras/EstadisticaDeConsumoDeElectricidadPorSector1990-2014}
	\caption{Estad�stica de consumo de electricidad por sector(1990-2014)}
	\label{fig:EstadisticaDeConsumoDeElectricidadPorSector1990-2014}
\end{figure}}

En el segundo tab, titulado ``Facturaci�n``, tenemos datos de facturaciones por grupo de consumidores.  Al igual que el gr�fico anterior observamos que la zona residencial es a la que m�s facturas se emiten.


\textsc{\begin{figure}[H]
	\centering
	\includegraphics[width=\linewidth]{figuras/ImporteFacturadoPorAnoYSector1990-2014}
	\caption{Importe facturado por a�o y sector(1990-2014)}
	\label{fig:ImporteFacturadoPorAnoYSector1990-2014}
\end{figure}}
\subsubsection{Panel comparativo de Tasa de crecimiento y el Consumo de energ�a}

En la figura ~\ref{fig:TasaDeCrecimientoVsConsumoDeEnergia} se presenta un panel comparativo entre la tasa de crecimiento de clientes y consumo de energ�a. Tambi�n se cuenta con un mapa para filtrar por regi�n del pa�s. 

\textsc{\begin{figure}[H]
	\centering
	\includegraphics[width=\linewidth]{figuras/TasaDeCrecimientoVsConsumoDeEnergia}
	\caption{Tasa de crecimiento vs Consumo de energ�a}
	\label{fig:TasaDeCrecimientoVsConsumoDeEnergia}
\end{figure}}

En el gr�fico situado a la derecha del mapa (~\ref{fig:TasaDeCrecimientoVsConsumoDeEnergia2}), se tiene el consumo y crecimiento de clientes. Para un mejor analisis fue necesario suavizar los datos calculando lineas de tendencia, debido a una inestabilidad de los datos de consumo. As� se puede apreciar, que existe una tendencia de crecimiento sostenido durante el tiempo de clientes. Sin embargo, la tendencia que el consumo crezca es mayor al de clientes.

\textsc{\begin{figure}[H]
	\centering
	\includegraphics[width=\linewidth]{figuras/TasaDeCrecimientoVsConsumoDeEnergia2}
	\caption{Tasa de crecimiento vs Consumo de energ�a, filtrado por el departamento Alto Paran�}
	\label{fig:TasaDeCrecimientoVsConsumoDeEnergia2}
\end{figure}}

Debajo se presenta un cuadro con la l�nea de cada valor (Figura ~\ref{fig:ProyeccionDeClientesYConsumosParaLosProximos5Anos}), de crecimiento y consumo. Con el uso de un recurso disponible que cuenta la herramienta escogida Tableau, es posible realizar an�lisis predictivo  (forecasting) del crecimiento y consumo. Tableau utiliza un algoritmo llamado Suavizado Exponencial, muy conocido en el �rea de Matem�ticas Estad�sticas. En este gr�fico se puede notar que existe una mayor probabilidad que en los pr�ximos a�os aumente considerablemente el consumo, superando su media. Sin embargo, se nota que el ritmo de crecimiento de clientes es sostenible, y no tiene una alta probabilidad de sufrir un aumento abrupto.

\textsc{\begin{figure}[H]
		\centering
		\includegraphics[width=\linewidth]{figuras/ProyeccionDeClientesYConsumosParaLosProximos5Anos}
		\caption{Proyecci�n de clientes y consumos para los pr�ximos 5 a�os}
		\label{fig:ProyeccionDeClientesYConsumosParaLosProximos5Anos}
	\end{figure}}
	
\subsubsection{Panel de Tasa de crecimiento poblacional y Consumo de energ�a anual}

\textsc{\begin{figure}[H]
	\centering
	\includegraphics[width=\linewidth]{figuras/TasaDeCrecimientoPoblacionalYConsumoDeEnergiaAnoTrasAno}
	\caption{Panel de Tasa de crecimiento poblacional y Consumo de energ�a anual}
	\label{fig:TasaDeCrecimientoPoblacionalYConsumoDeEnergiaAnoTrasAno}
\end{figure}}

En la figura~\ref{fig:TasaDeCrecimientoPoblacionalYConsumoDeEnergiaAnoTrasAnoMapa} se presenta el panel comparativo de datos de crecimiento poblacional de la DGEEC y de consumo de la ANDE. En el mapa, cuando el color es m�s oscuro el consumo de energ�a es mayor, y si el color es m�s claro, el consumo es menor. Se puede notar que los departamentos Central y Alto Paran� son los que tiene mayor consumo de energ�a. 
Este tipo de gr�fico es muy �til cuando se desea analizar informaci�n de forma general y georeferenciada. Al ubicar el mouse sobre cualquier departamento, se presenta una ventana emergente indicando el valor de consumo del departamento seleccionado. Al seleccionar un departamento del mapa, los dem�s gr�ficos tambi�n se actualizan filtrando el departamento seleccionando. 

\textsc{\begin{figure}[H]
	\centering
	\includegraphics[width=\linewidth]{figuras/TasaDeCrecimientoPoblacionalYConsumoDeEnergiaAnoTrasAnoMapa}
	\caption{Consumo por departamento}
	\label{fig:TasaDeCrecimientoPoblacionalYConsumoDeEnergiaAnoTrasAnoMapa}
\end{figure}}
\noindent
 Los gr�ficos situados a la derecha y debajo del mapa presentan la tasa de crecimiento poblacional y de consumo de energ�a. Es importante notar que la tasa de crecimiento de la poblaci�n es casi constante, esto es, no tiene una gran variaci�n en el tiempo, ni una tendencia a alejarse de la media. Sin embargo, la tasa de crecimiento de consumo tiene una l�nea de tendencia a crecer durante el tiempo. As� puede apreciarse que, el consumo de energ�a no tiene una relaci�n de proporcionalidad con respecto al crecimiento de la poblaci�n.

\textsc{\begin{figure}[H]
	\centering
	\includegraphics[width=\linewidth]{figuras/ProyeccionDeCrecimientoPoblacionalYConsumoDeEnergia}
	\caption{Proyecci�n de crecimiento poblacional y consumo ee energ�a}
	\label{fig:ProyeccionDeCrecimientoPoblacionalYConsumoDeEnergia}
\end{figure}}
\noindent
En este gr�fico,  se muestra la misma informaci�n que el gr�fico anterior pero con diferente perspectiva, en este caso se calcula el porcentaje de crecimiento anual tanto de la poblaci�n, as� como del consumo.


\textsc{\begin{figure}[H]
	\centering
	\includegraphics[width=\linewidth]{figuras/ProyeccionDeClientesYConsumosParaLosProximos5Anos2}
	\caption{Tasa de crecimiento poblacional y consumo de energ�a a�os tras a�os}
	\label{fig:ProyeccionDeClientesYConsumosParaLosProximos5Anos2}
\end{figure}}

		\chapter{CAPITULO 4}
\section{Conclusiones y Trabajos futuros}
		
		\renewcommand{\bibname}{Bibliograf�a}
		\bibliographystyle{plain}
		\bibliography{bibliografia/bibFile}
	
\end{document}