 % !TeX document-id = {691fd987-4a4c-464c-b0de-b4bb28391dc9}
% !TeX program = PdfLaTeX
% !TeX encoding = ISO-8859-1
% !TeX spellcheck = es_ES



	
\documentclass[letterpaper,12pt]{book}
	
	%identation on first paragraph
	%\usepackage{indentfirst}
	\setlength\parindent{1.25cm}
	%Align to left all text
	%\parfillskip=2em plus 0.1\textwidth
	%\unskip\parfillskip 0pt \par
	
	%identation in specific text block with addmargin command
	\usepackage{scrextend}
	
	%make links automatically
	%\usepackage{hyperref}
	
	\usepackage{ae}
	\usepackage[T1]{fontenc}
	\usepackage[left=2.54cm,top=2.54cm,right=2.54cm,bottom=2.54cm]{geometry} 
	\usepackage{graphicx} 
	\usepackage{wrapfig}
	\graphicspath{ {imagenes/} } 
	\usepackage{fancyhdr} 
	\usepackage{sectsty}
	\usepackage{tocloft}
	\usepackage[nopostdot,style=super,nonumberlist,toc]{glossaries}
	\usepackage{float}
	%\restylefloat{table}
	
	%\usepackage[center]{titlesec}
	
	\usepackage[spanish]{babel}
	\usepackage{apacite}
	\bibliographystyle{apacite}
	
	\usepackage{url}
	\usepackage{enumitem}
	%to external document referencing
	\usepackage{xr}
	%add source to figures
	\usepackage{caption}
	%Para usar interlineado especifico en algunas partes
	\usepackage{setspace}
	
	\usepackage{ragged2e}
	
	%\titleformat{\chapter}[display]
	%{\normalfont\bfseries\centering}{}{0pt}{\Large}

\makeatletter
\def\@makechapterhead#1{%
	\vspace*{50\p@}%
	{\parindent \z@ \raggedright \normalfont
		\ifnum \c@secnumdepth >\m@ne
		\if@mainmatter
		%\huge\bfseries \@chapapp\space \thechapter
		\Huge\bfseries \centering \thechapter.\space%
		%\par\nobreak
		%\vskip 20\p@
		\fi
		\fi
		\interlinepenalty\@M
		\Huge \bfseries #1\par\nobreak
		\vskip 40\p@
	}}
\makeatother
	
%\makeatletter
%\def\@makechapterhead#1{%
%\vspace*{50\p@}%
%{\parindent \z@ \raggedright \normalfont
%	\ifnum \c@secnumdepth >\m@ne
%	\centering\huge\bfseries \@chapapp\space \thechapter
%	\par\nobreak
%	\vskip 20\p@
%	\fi
%	\interlinepenalty\@M
%	\Huge \bfseries #1\par\nobreak
%	\vskip 40\p@
%}}
%\def\@makeschapterhead#1{%
%\vspace*{50\p@}%
%{\parindent \z@ \raggedright
%	\normalfont
%	\interlinepenalty\@M
%	\centering\Huge \bfseries  #1\par\nobreak
%	\vskip 40\p@
%}}
%\renewcommand\section{\@startsection {section}{1}{\z@}%
%	{-3.5ex \@plus -1ex \@minus -.2ex}%
%	{2.3ex \@plus.2ex}%
%	{\centering\normalfont\Large\bfseries}}
%\makeatother


	
	%formula caption
	\DeclareCaptionType{mycapequ}[][List of equations]
	\captionsetup[mycapequ]{labelformat=empty}
	
	%\titleformat{\section}
	%{\normalfont\Large\bfseries}
	%{\thesection}{1em}{}

	%Centra el titulo de los capitulos
	%\chapterfont{\centering}
	%Centra el titulo de las secciones
	%\sectionfont{\centering}
	%Centra el titulo de las sub-secciones
	%\subsectionfont{\centering}
	
	

	
	%Usar cuando queremos ver el area de impresion(muy bueno)
	%\usepackage{showframe}
	
	
	%Configura el footer con numeros de pagina CE=CENTER EVENT, CO=CENTER ODD
	%%\pagestyle{fancy}	
	%%\fancyhf{}
	%%\fancyfoot[CE,CO]{\ifnum\value{page}<17\relax\else\thepage\fi}
	%elimina el subrayado del header
	%%\renewcommand{\headrulewidth}{0pt}
	%muestra un subrayado despues del footer
	%\renewcommand{\footrulewidth}{0.4pt}
	
	
	% Page style for preliminary pages.
	\fancypagestyle{preliminary}{
		\fancyhf{}% Clear header/footer
		\fancyfoot[C]{\thepage}% Footer
		\renewcommand{\headrulewidth}{0pt}% No header rule
	}
	
	% Page style for main matter.
	\fancypagestyle{mainmatter}{
		\fancyhf{}% Clear header/footer
		\fancyfoot[C]{\thepage}% Footer
		\renewcommand{\headrulewidth}{0pt}% No header rule
	}
	
	
	% Glosario
	%\usepackage[doublespacing]{setspace}
	\setlength{\glsdescwidth}{0.9\linewidth}
	\newglossarystyle{mylong}{
		\glossarystyle{long}
		\renewcommand*{\glossaryentryfield}[5]{%
			\glsentryitem{##1}\glstarget{##1}{##2} & ........................ ##3\glspostdescription\space ##5\\[8pt]}%
		\renewcommand{\glsgroupskip}{}
	}
	\makeglossaries% Inicia
	%\renewcommand{\glstextformat}[1]{\textnormal{#1}}
	%\renewcommand*{\glsdisplayfirst}[4]{\textnormal{#1#4}}
	%\renewcommand{\glstextformat}[1]{\textit{\textnormal{#1}}}
	%\renewcommand*{\glsdisplayfirst}[4]{\textbf{#1#4}}
	%\renewcommand{\glstextformat}[1]{\textbf{#1}}

	%headers spacing
	%\titlespacing{\section}{0pt}{*0}{*0}
	%\titlespacing{\subsection}{0pt}{*0}{*0}
	%\titlespacing{\subsubsection}{0pt}{*0}{*0}
	%\everymath{\displaystyle}
	\DeclareMathSizes{12}{20}{14}{10}
	
	%include files with figure declaration%
\externaldocument{figura1}
\externaldocument{figura2}
\externaldocument{figura3}
\externaldocument{figura4}
\externaldocument{figura5}
\externaldocument{figura6}
\externaldocument{figura7}
\externaldocument{figura8}
\externaldocument{figura9}
\externaldocument{figura10}
\externaldocument{figura11}
\externaldocument{figura12}
\externaldocument{figura13}
\externaldocument{figura14}
\externaldocument{figura15}
\externaldocument{figura16}
\externaldocument{figura17}
\externaldocument{figura18}
\externaldocument{figura19}
\externaldocument{figura20}
\externaldocument{figura21}
\externaldocument{figura22}
\externaldocument{figura23}
\externaldocument{figura24}
\externaldocument{figura25}
\externaldocument{figura26}
\externaldocument{figura27}
\externaldocument{figura28}

	%interlineado
	\renewcommand{\baselinestretch}{2}	
	
	%list of abbreviations
	%\usepackage[intoc]{nomencl}
	%\makenomenclature
	
	\makeindex % <=== !!!!!	
		
	\usepackage{sectsty}
	%Center chapter
	\chapterfont{\centering}
	%Center section
	\sectionfont{\centering}
	%\subsectionfont{\centering}
\begin{document}
		%bibliografia abreviatura		
		\renewcommand{\BOthers}[1]{et al.\hbox{}}%
		
		
		%elimina la justificacion de texto vertical
		\raggedbottom
		
		\pagenumbering{gobble}
		%Cubiertas o tapas del libro
		\include{cubierta_cubierta-1}
		%\clearpage\null		
		\include{cubierta_cubierta-2}
		
		\pagestyle{preliminary}\pagenumbering{roman}
		\fancypagestyle{plain}{\pagestyle{preliminary}}% Correct plain page style
		\setcounter{page}{2}
		%\clearpage\null
		\begin{center}	
	\textbf{
	\input{globales_autor1}
	\\
	\input{globales_autor2}
	}
\end{center}




\vfill

\hspace{.25\textwidth} % posicionando a minipage
\begin{minipage}{.8\textwidth}\setstretch{1.0}
	``DATA DISCOVERY APLICADOS A DATOS DEL PARAGUAY``. Proyecto de Fin de Carrera presentado como requisito parcial para optar al t�tulo de Licenciado en An�lisis de Sistemas.
	

	Facultad de Ciencias y Tecnolog�a, Universidad Cat�lica ``Nuestra Se�ora de la Asunci�n"\\
	
	Tutor: Ing. Ricardo Luis Brunelli Montero		
\end{minipage}



%llena el espacio vacio
\vfill
\begin{center}
	Hernandar�as, octubre de 2016
\end{center}



		%\clearpage\null
		\include{cubierta_cubierta-4}
		\begin{center}	
	\textbf{
	\input{globales_autor1}
	\\
	\input{globales_autor2}
	}
\end{center}

\vspace*{1\baselineskip}
%\begin{center}
	\textbf{DATA DISCOVERY APLICADOS A DATOS DEL PARAGUAY}
\end{center}


\vspace*{1\baselineskip}
\hspace{.1\textwidth} % posicionando a minipage
\begin{minipage}{.8\textwidth}\setstretch{1.0}
	``DATA DISCOVERY APLICADOS A DATOS DEL PARAGUAY``. Proyecto de Fin de Carrera presentado como requisito parcial para optar al t�tulo de Licenciado en An�lisis de Sistemas.
	
\end{minipage}

\vspace*{1\baselineskip}
\begin{center}\setstretch{1.0}
	Mesa Examinadora\\
	
	\vspace*{1\baselineskip}
	\makebox[3in]{\hrulefill}
	\par\noindent
	Prof. Ing. Juan Carlos Ocampos N��ez\\
	Presidente de Mesa\\
	\vspace*{1\baselineskip}
	\makebox[3in]{\hrulefill}
	\par\noindent
	Prof. Lic. Miguel Angel Duarte\\	
	Profesor Testigo\\
	\vspace*{1\baselineskip}
	\makebox[3in]{\hrulefill}
	\par\noindent
	Prof. Ing. Eduardo Mar�a Morel Meyer\\
	Profesor Testigo\\
\end{center}
\vspace*{1\baselineskip}
Aprobado en fecha:




%llena el espacio vacio
\vfill
\begin{center}
	Hernandar�as, octubre de 2016
\end{center}



%\vspace*{1\baselineskip}
%\hspace{.5\textwidth} % posicionando a minipage

				
		%Lista de contenido
		\pagebreak				
		\renewcommand{\contentsname}{\hfill �NDICE \hfill}
		\tableofcontents
		
		%Lista de Figuras
		\pagebreak		
		\newlength{\mylen}		
		\renewcommand{\listfigurename}{\hfill�NDICE DE FIGURAS\hfill}
		\renewcommand{\figurename}{Figura}
		\renewcommand{\cftfigpresnum}{Figura\enspace}
		\renewcommand{\cftfigaftersnum}{:}
		\settowidth{\mylen}{\cftfigpresnum\cftfigaftersnum}
		\addtolength{\cftfignumwidth}{\mylen}
		\listoffigures
		
		%Lista de Tablas
		\pagebreak		
		\renewcommand{\listtablename}{\hfill LISTA DE TABLAS\hfill}
		\listoftables

		%GLOSARIO		
		\pagebreak
		\newacronym{sig:BI}{BI}{\textit{Inteligencia de Negocios}}
\newacronym{sig:ANDE}{ANDE}{\textit{Administraci�n Nacional de Electricidad}}
\newacronym{sig:DGEEC}{DGEEC}{\textit{Direcci�n General de Estad�sticas, Encuestas y Censo}}
\newacronym{sig:CPU}{CPU}{\textit{Unidad Central de Procesos}}
\newacronym{sig:ERP}{ERP}{\textit{Sistemas de Planificaci�n de Recursos Empresariales}}
\newacronym{sig:CRM}{CRM}{\textit{Administraci�n basada en la relaci�n con los clientes}}
\newacronym{sig:OLAP}{OLAP}{\textit{Procesamiento anal�tico en l�nea}}
\newacronym{sig:UX}{UX}{\textit{Experiencia del Usuario}}
\newacronym{sig:Dashboard}{Dashboard}{\textit{Paneles indicadores}}
\newacronym{sig:DSS}{DSS}{\textit{Sistema de apoyo a las decisiones}}
\newacronym{sig:Datawarehouse}{Datawarehouse}{\textit{Almac�n de datos}}
\newacronym{sig:TI}{TI}{\textit{Tecnolog�a de la informaci�n}}
\newacronym{sig:SaaS}{SaaS}{\textit{Software como servicio}}




		\printglossary[style=mylong,title=Lista de Siglas y Acr�nimos]
		
		\begin{center}
		\fontsize{16}{16}\selectfont
	\textbf{DEDICATORIA}
\end{center}

\vfill
\begin{minipage}{.8\textwidth}
	
		A mis ya difuntos abuelos: Eladio Segovia y Rosa Rodriguez  por sus sacrificios durante mi etapa de ni�ez y adolescencia. Por la perseverancia y la creencia de que con una buena educaci�n todo se puede cambiar, por los ejemplos de superaci�n y lucha incansable, por los consejos que siempre me dieron para sobrellevar los desaf�os en la vida. A mi padre Eulogio C�ceres que siempre me apoy� y no dudo que siempre me seguir� apoyando en todo momento. A mi esposa Nilda Vera y mi hija Raquel C�ceres que tambi�n han participado de forma impl�cita, cedi�ndome espacio y tiempo durante las horas que me han tocado trabajar para concluir este proyecto.
	
	\textbf{\begin{flushright}\input{globales_autor1}\end{flushright}}
\end{minipage}

\vspace*{3\baselineskip}
\begin{minipage}{.8\textwidth}
	
	Gracias a esas personas importantes en mi vida, que siempre estuvieron listas para brindarme toda su ayuda, ahora me toca regresar un poquito de todo lo inmenso que me han otorgado.
	
	\textbf{\begin{flushright}\input{globales_autor2}\end{flushright}}
\end{minipage}

		\section*{Agradecimientos}\setstretch{1.0}


	\vspace*{3\baselineskip}

	\begin{addmargin}[1em]{1em}% 1em left, 1em right
		
		A Dios por la fortaleza que siempre nos ha dado en todos los momentos de nuestra vida.\\
		
		\noindent Gracias, a nuestro tutor, el Ing. Ricardo Luis Brunelli. Gracias por su paciencia, dedicaci�n, motivaci�n, criterio y aliento. Ha hecho f�cil lo dif�cil. Ha sido un privilegio poder contar con su gu�a y ayuda.\\
		
		\noindent A nuestra familia quienes por ellos somos lo que somos. A nuestros padres por su apoyo, consejos, comprensi�n, amor, ayuda en los momentos dif�ciles, y por ayudarnos con los recursos necesarios para estudiar. Nos han dado todo lo que somos como persona, valores, principios, car�cter, empe�o, perseverancia, y coraje para conseguir nuestros objetivos.
	\end{addmargin}
	
	\vspace*{1\baselineskip}
	\textbf{\begin{flushright}\input{globales_autor1}\end{flushright}}	
		
		\chapter*{\centering RESUMEN}
	%Se realiz� una investigaci�n en el �rea de la Inteligencia de %Negocios (BI) con datos de la instituci�n p�blica ANDE con la %finalidad de estimar un potencial crecimiento de su estructura de %acuerdo al crecimiento de la poblaci�n pudiendo as� tener un %mejor entendimiento de los datos y facilitar la toma de %decisiones.
Hasta hace poco tiempo, la mayor�a de las organizaciones prove�an datos estructurados, limpios, e integrados, resumidos a niveles convenientes para plataformas convencionales. Data Warehouse e \gls{sig:BI} dominaban ese enfoque. Otras organizaciones, principalmente aquellas centradas en internet, desarrollaron algunas alternativas para gestionar y analizar grandes vol�menes de datos directamente de sus sitios y aplicaciones web, hoy generalmente denominado Big Data. Aquellos datos obtenidos, en su mayor�a, eran heterog�neos y hasta inclusive no estructurados, y esa situaci�n gener� la necesidad de crear otro tipo de herramienta que ayude al tomador de decisi�n en la b�squeda de patrones y relaciones. Este nuevo enfoque, denominado Data Discovery, no pod�a ser igual a las t�cnicas ya tradicionales, adem�s deb�a tener caracter�sticas como innovaci�n visual, facilidad de uso, \gls{sig:UX} para que se asemeje a un BI guiado por un usuario experto del negocio. En este trabajo se presenta una propuesta del estado del arte del �rea de BI y espec�ficamente Data Discovery. Se aplican estas t�cnicas a datos de dos instituciones del estado, demostrando los beneficios de aplicar este tipo de t�cnica.

\vspace*{1\baselineskip}
\textbf{Palabras clave:} Data Discovery, Business Intelligence, Data Warehouse.
		\vspace*{1\baselineskip}

\chapter*{\centering ABSTRACT}
	Until recent time, most organizations provide structured, clean, and integrated data, summarized in desirable levels for conventional platforms. Data Warehouse and BI (Business intelligence) ruled that approach. Other organizations, mainly those focusing on the Internet, developed some alternatives to manage and analyze large volumes of data directly from their websites and web applications, now generally called Big Data. Those data, mostly, were heterogeneous and even including unstructured, and this situation generated the need to create another type of tool that helps the decision maker in the search for patterns and relationships. This new approach, called Data Discovery, could not be equal to the traditional techniques also should have features like visual innovation, usability, UX (User Experience) so that it resembles a BI guided by an expert business user. This paper presents a proposal of state of the art area BI and Data Discovery is presented specifically. These techniques are applied to data of two state institutions, demonstrating the benefits of applying this kind of technique.


\vspace*{1\baselineskip}
\textbf{Keywords:} Data Discovery, Business Intelligence, Data Warehouse.
		
		\title{Data Discovery Paraguay}
		\author{Ariel Hern�n Landaida Duarte, Iv�n Ariel Caceres Ca�ete}
		%oculta la fecha por defecto
		\date{}
		
		%\maketitle		
		%Data Discovery	  

		
		\mainmatter
		\pagestyle{mainmatter}\pagenumbering{arabic}
		\fancypagestyle{plain}{\pagestyle{mainmatter}}% Correct plain page style
	  
		
		%Oculta la palabra Chapter
		%\renewcommand{\chaptername}{}
		%Oculta la numeraci�n de los chapters
		%\renewcommand{\thechapter}{}
		
		

		%Insertar capitulos		
		\begingroup
			\renewcommand{\cleardoublepage}{}
			\renewcommand{\clearpage}{}
			\chapter{Introducci�n}
En este trabajo fue elaborada una propuesta de estado del arte del �rea de BI (Business intelligence) y espec�ficamente Data Discovery. El objetivo fue aplicar los conocimientos adquiridos al entorno nacional, en este caso aplicados a una instituci�n del estado (Administraci�n Nacional de Electricidad, ANDE), para el apoyo en la toma de decisiones.
Aplicaremos las t�cnicas de Data Discovery a los datos de dos instituciones del estado, espec�ficamente la \gls{sig:ANDE} y \gls{sig:DGEEC}, donde demostraremos que con datos de calidad podr�amos detectar oportunidades que nos faciliten la toma de decisiones en la instituci�n. Utilizaremos conjuntos de datos de las instituciones mencionadas m�s arriba para este fin.
Este trabajo est� estructurado de la siguiente forma:	
El cap�tulo I presenta el planteamiento del problema, los objetivos generales y espec�ficos, concluyendo con la justificaci�n.
El cap�tulo II presenta el marco te�rico. Se compone de conceptos, componentes, tecnolog�as de BI, y el framework de Gartner.
El cap�tulo III presenta los diferentes paneles indicadores (dashboards) realizados con la herramienta seleccionada para aplicar data discovery sobre los conjuntos de datos existentes. Se explica el cuadrante m�gico de Gartner, el cual apoy� a la selecci�n de la herramienta mencionada. Adem�s se presentan ejemplos de an�lisis de datos y la informaci�n obtenida, como resultado de la combinaci�n del uso de las herramientas y datos.\\
			\pagebreak
			\chapter{Planteamiento del problema}
Actualmente la mayor�a de las instituciones p�blicas e incluso las del sector privado, tienen un bajo nivel de inversi�n en tecnolog�a. Generalmente en empresas que exigen toma de decisi�n, con frecuencia optan por decisiones de negocios no �ptimas debido a que no poseen la suficiente experiencia o suficiente datos procesados del negocio para llegar al correcto an�lisis, o pueden estar usando herramientas incorrectas. Teniendo �nicamente la experiencia como herramienta, puede ser suficiente s�lo en organizaciones peque�as, donde se adquiere conocimiento sin necesidad de alguna herramienta de an�lisis de datos, por ejemplo, sabr�amos cuales son los productos m�s vendidos o m�s rentables en un negocio peque�o, y esto se complica a medida que la organizaci�n crece, cuando la cantidad de sucursales y variedad de productos es amplia, se pierde el control sin ayuda de estas herramientas.

La calidad del conocimiento se basa principalmente en la calidad de la informaci�n. Estas informaciones son obtenidas a trav�s de un an�lisis profundo de datos, por consiguiente estos datos tambi�n deben ser de buena calidad. Si una organizaci�n posee la capacidad de obtener lo que desea por medio de los datos, es seguro que su crecimiento ser� positivo, debido a que tomar� mejores decisiones, y estos ayudar�an a su evoluci�n y estabilidad a lo largo del tiempo.

Existen ocasiones en que la organizaci�n posee suficientes datos pero no consigue analizarlos o procesarlos por falta de conocimiento del negocio, adem�s de desconocimiento t�cnico de las herramientas por los cu�les surgen las siguientes preguntas.

\section{Pregunta General}
?`Qu� resultados obtendr�amos usando las t�cnicas de Data Discovery?

\section{Preguntas Espec�ficas}
\begin{itemize}

\item ?`C�mo se encuentran actualmente las tecnolog�as de \gls{sig:BI} y Data discovery?
\item ?`Cu�l es la herramienta m�s adecuada para este trabajo?
\item ?`Qu� podr�amos descubrir al analizar y cruzar datos?
\item ?`Ser� posible reflejar el an�lisis de los datos con la herramienta seleccionada?
\item ?`Podr�an ser elaborados gr�ficos de tendencias y proyecciones? 
\item ?`Qu� es posible demostrar con los datos?

\end{itemize}





			\pagebreak
			\chapter{Objetivos}

\section{Objetivo General}
Aplicar las t�cnicas de Data Discovery a datos de dos instituciones del estado: \gls{sig:ANDE} y \gls{sig:DGEEC}.

\section{Objetivos Espec�ficos}
\begin{itemize}

	\item Estudio del estado del arte de \gls{sig:BI} y Data discovery.
	\item Selecci�n de la herramienta adecuada para este trabajo.
	\item An�lisis y cruzamiento de datos para el descubrimiento de situaciones de inter�s.
	\item Elaboraci�n de \gls{sig:Dashboard} que reflejen el an�lisis de los datos con la herramienta seleccionada.
	\item Elaboraci�n de gr�ficos de tendencias, proyecciones de consumo y clientes de la instituci�n. 
	\item Proponer algunos \gls{sig:Dashboard} para demostrar que con los datos prove�dos es posible detectar situaciones, y estimar o visualizar acontecimientos de inter�s para una organizaci�n
\end{itemize}
			\pagebreak
			\chapter{Justificaci�n}
Falta de un ambiente anal�tico corporativo, que proporcione informaciones e indicadores necesarios para la toma de decisi�n.
Falta de agilidad para elaborar informes con indicadores de tendencias de consumo de energ�a, que apoye a la planificaci�n territorial de expansi�n de la red de transmisi�n el�ctrica.
Falta de indicadores de consumo geogr�fico de electricidad, relacionados con indicadores poblacionales, para apoyo en la planificaci�n de inversi�n en el aumento de capacidad de transformadores por regi�n.
			\input{contenido_marco_teorico}		
			\pagebreak
\fontsize{12}{12}\selectfont
%\vspace{-25mm}
\section{Selecci�n de la herramienta para Data Discovery}
%\vspace{-25mm}
Fueron analizados los estudios de Gartner\cite{herschel2015magic}\cite{sallam2015critical} para la selecci�n de la mejor herramienta que se adecue a los criterios necesarios para ser utilizado en esta tesis. Estos documentos realizan un an�lisis de las mejores herramientas del mercado, en un �rea de conocimiento. A continuaci�n se presenta el Cuadrante M�gico de Gartner, para herramientas de BI y Analytics.


%\vspace{-40mm}
\section{Cuadrante M�gico de Gartner}

\textsc{\begin{figure}[H]
	\centering
	\caption{Cuadrante M�gico para BI y Plataformas Anal�ticas}
	\includegraphics[width=160mm]{Magic-Quadrant.png}
	\caption*{Fuente: \cite{herschel2015magic}}
	\label{fig:magicQuadrant}
\end{figure}}

%\subsection{Tableau}
%Tableau tiene una posici�n fuerte en capacidad de ejecuci�n en el eje de l�deres del cuadrante. Esta herramienta fue la que mejor se adecu� a las necesidades del trabajo de Tesis, dado que cuenta con una versi�n p�blica para la construcci�n y publicaci�n de dashboards, adem�s de la facilidad de uso que nos proporciona. Tableau Desktop, la cual se basa en tecnolog�a drag and drop (arrastrar y soltar) permite analizar datos r�pidamente y permite ver los cambios en tiempo real sin necesidad de codificaci�n, de esta manera, posibilita a un usuario con no muchos conocimientos t�cnicos poder utilizarlo con mayor facilidad.

\noindent
Los l�deres del mercado se encuentran siempre en el cuadrante superior derecho. Se puede observar una amplia diferencia entre Tableau y los dem�s l�deres.\\
En las siguientes figuras se presenta el an�lisis de Gartner que eval�a las capacidades cr�ticas que debe tener una herramienta de BI y Analytics, para adecuarse a las necesidades del mercado.

\textsc{\begin{figure}[H]
	\centering
	\caption{Puntuaciones de producto o servicio para an�lisis descentralizado }
	\includegraphics[width=160mm]{productOrServiceScoresForDecentralizedAnalytics.png}
	\caption*{Fuente: \cite{parenteau2015rise}}
	\label{fig:productOrServiceScoresForDecentralizedAnalytics}
\end{figure}}

\textsc{\begin{figure}[H]
	\centering
	\caption{Puntuaciones de producto o servicio guiados por Data Discovery}
	\includegraphics[width=160mm]{productOrServiceScoresForGovernedDataDiscovery.png}
	\caption*{Fuente: \cite{parenteau2015rise}}
	\label{fig:productOrServiceScoresForGovernedDataDiscovery}
\end{figure}}

\noindent
Los valores posibles van del 1 al 5, conforme la siguiente evaluaci�n:

\begin{enumerate}
	\item Pobre o ausente: la mayor�a de los requisitos de esta capacidad no fueron alcanzadas.
	\item Justo: Algunos de los requisitos fueron alcanzados.
	\item Bueno: cumple con los requisitos.
	\item Excelente: alcanza o excede algunos requisitos.
	\item Superior: excede significativamente los requisitos.
\end{enumerate}
\noindent
Tableau tiene una posici�n fuerte en capacidad de ejecuci�n (producto/servicio, su oferta, ejecuci�n de ventas, marketing, experiencia del cliente) en el eje de l�deres del cuadrante. Esta herramienta fue la que mejor se adecu� a las necesidades del trabajo de Tesis, dado que cuenta con una versi�n p�blica para la construcci�n y publicaci�n de dashboards, adem�s de la facilidad de uso que nos proporciona. Tableau Desktop, la cual se basa en tecnolog�a drag and drop (arrastrar y soltar) permite analizar datos r�pidamente y permite ver los cambios en tiempo real sin necesidad de codificaci�n, de esta manera, posibilita a un usuario con escasos conocimientos t�cnicos, poder utilizarlo de igual manera.

\textsc{\begin{figure}[H]
	\centering
	\caption{Arrastre el campo pa�s para el campo desplegable se�alado.}
	\includegraphics[width=0.7\linewidth]{imagenes/tableauDragAndDrop2}
	\caption*{Fuente: \cite{peck2013tableau}}
	\label{fig:tableauDragAndDrop2}
\end{figure}}

\textsc{\begin{figure}[H]
	\centering
	\caption{Arrastrar hojas de trabajo al dashboard.}
	\includegraphics[width=0.7\linewidth]{imagenes/tableauDragAndDrop1}
	\caption*{Fuente: \cite{peck2013tableau}}
	\label{fig:tableauDragAndDrop1}
\end{figure}}

\noindent
De una forma �gil el usuario puede conectarse a diversas fuentes de datos y crear paneles interactivos, conectando entre s� los diferentes componentes (tipo de gr�fico) que proporciona la herramienta. La herramienta permite utilizar componentes como filtros, siendo o no de la misma fuente de datos siempre y cuando los datos coincidan en los diversos conjuntos.
Pueden ser utilizadas en la organizaciones para comprender r�pidamente diferentes aspectos del negocio. Tambi�n se puede utilizar para realizar proyecciones o tendencias,  la cual Tableau nos ofrece de manera autom�tica.

			\pagebreak
\chapter{Marco Metodol�gico}
\section{Alcance}
Aplicamos las t�cnicas de Data Discovery a los datos de dos instituciones del estado, espec�ficamente la \gls{sig:ANDE} y \gls{sig:DGEEC}, donde mostramos que con datos de calidad se pueden detectar oportunidades que nos facilitan la toma de decisiones en la instituci�n. Utilizamos conjuntos de datos de las instituciones mencionadas m�s arriba para este fin.
\section{Enfoque}	
El enfoque que utilizamos es el cuantitativo.\\
\noindent (...) que por lo com�n, usa la recolecci�n y el an�lisis de datos para contestar preguntas de investigaci�n y probar hip�tesis establecidas previamente, y conf�a en la medici�n num�rica, el conteo, y en el uso de la estad�stica para intentar establecer con exactitud  patrones en una poblaci�n. \cite{monge2010estudio}
\section{T�cnica e Instrumentos de recolecci�n de datos}
La t�cnica aplicada en este trabajo en la recolecci�n de datos fue la investigaci�n de documentos cient�ficos procedentes de publicaciones de empresas pioneras en Data Discovery y de expertos en el �rea.

			\input{contenido_resultados}
			\pagebreak
\chapter{Conclusiones}
	A lo largo de este trabajo demostramos c�mo es posible dar valor agregado a los datos y en base a estos llegar a tomar decisiones que proporcionen mayores beneficios.\\

	\noindent Teniendo en cuenta los resultados obtenidos se pueden observar las siguientes cuestiones:

\begin{enumerate}
	\item La tasa de crecimiento del consumo es mayor a la tasa de aumento de clientes.
	\item Debido a este aumento en el consumo de energ�a del pa�s, disminuy� la cantidad de energ�a exportada.
	\item Aunque se tuvo una disminuci�n en la energ�a exportada, se obtuvo un crecimiento en el valor facturado. Esto demuestra una mejor�a en el precio de venta de la energ�a al Brasil (Itaip�) o Argentina (Yacyret�).
	\item Seg�n el pron�stico de crecimiento para los pr�ximos a�os, el consumo de energ�a tendr� un crecimiento mayor al de la cantidad de clientes, y la poblaci�n del pa�s.
	\item La tasa de crecimiento de la poblaci�n se mantiene constante durante el tiempo, esto es, la poblaci�n crece a una tasa sostenida. Sin embargo, la tasa de aumento en el consumo de energ�a es considerablemente mayor, y demuestra un aumento abrupto para los pr�ximos a�os, independiente a la tasa de aumento de la poblaci�n y de nuevos clientes de la ANDE.
\end{enumerate}
\noindent
Teniendo en cuenta estas cuestiones, este trabajo puede servir de herramienta para la planificaci�n en la inversi�n en la capacidad de transmisi�n y distribuci�n de electricidad dentro del territorio del pa�s, para los pr�ximos a�os, adem�s de ayudar con indicadores para montar una estrategia de exportaci�n para replantear las condiciones actuales de venta de la energ�a al exterior.\\
			\vfill
\pagebreak
\chapter{Trabajos futuros}
En la metodolog�a propuesta hemos cubierto s�lo algunas de las etapas de los procesos de \gls{sig:BI}. Como desaf�os a futuro se plantea aplicar \gls{sig:DM} a los datos para detectar nuevos patrones que servir�n de apoyo a la construcci�n de nuevos \gls{sig:Dashboard} y a la toma de decisiones.

Tambi�n se propone refinar los gr�ficos que utilizan geolocalizaci�n debido a que actualmente no se pueden referenciar con la granularidad deseada ya que solo podemos visualizar los departamentos, ser�a interesante agregar barrios, sub-estaciones con sus lineas de transmisi�n.			
										
		\endgroup

	
		
		\clearpage
		\renewcommand{\bibname}{Bibliograf�a}		
		\bibliography{bibFile}
		
	
\end{document}