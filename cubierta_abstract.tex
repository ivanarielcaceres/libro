\vspace*{1\baselineskip}

\chapter*{\centering ABSTRACT}
	Until recent time, most organizations provide structured, clean, and integrated data, summarized in desirable levels for conventional platforms. Data Warehouse and BI (Business intelligence) ruled that approach. Other organizations, mainly those focusing on the Internet, developed some alternatives to manage and analyze large volumes of data directly from their websites and web applications, now generally called Big Data. Those data, mostly, were heterogeneous and even including unstructured, and this situation generated the need to create another type of tool that helps the decision maker in the search for patterns and relationships. This new approach, called Data Discovery, could not be equal to the traditional techniques also should have features like visual innovation, usability, UX (User Experience) so that it resembles a BI guided by an expert business user. This paper presents a proposal of state of the art area BI and Data Discovery is presented specifically. These techniques are applied to data of two state institutions, demonstrating the benefits of applying this kind of technique.


\vspace*{1\baselineskip}
\textbf{Keywords:} Data Discovery, Business Intelligence, Data Warehouse.