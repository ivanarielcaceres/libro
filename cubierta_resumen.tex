\chapter*{\centering RESUMEN}
	%Se realiz� una investigaci�n en el �rea de la Inteligencia de %Negocios (BI) con datos de la instituci�n p�blica ANDE con la %finalidad de estimar un potencial crecimiento de su estructura de %acuerdo al crecimiento de la poblaci�n pudiendo as� tener un %mejor entendimiento de los datos y facilitar la toma de %decisiones.
Hasta hace poco tiempo, la mayor�a de las organizaciones prove�an datos estructurados, limpios, e integrados, resumidos a niveles convenientes para plataformas convencionales. Data Warehouse e \gls{sig:BI} dominaban ese enfoque. Otras organizaciones, principalmente aquellas centradas en internet, desarrollaron algunas alternativas para gestionar y analizar grandes vol�menes de datos directamente de sus sitios y aplicaciones web, hoy generalmente denominado Big Data. Aquellos datos obtenidos, en su mayor�a, eran heterog�neos y hasta inclusive no estructurados, y esa situaci�n gener� la necesidad de crear otro tipo de herramienta que ayude al tomador de decisi�n en la b�squeda de patrones y relaciones. Este nuevo enfoque, denominado Data Discovery, no pod�a ser igual a las t�cnicas ya tradicionales, adem�s deb�a tener caracter�sticas como innovaci�n visual, facilidad de uso, \gls{sig:UX} para que se asemeje a un BI guiado por un usuario experto del negocio. En este trabajo se presenta una propuesta del estado del arte del �rea de BI y espec�ficamente Data Discovery. Se aplican estas t�cnicas a datos de dos instituciones del estado, demostrando los beneficios de aplicar este tipo de t�cnica.

\vspace*{1\baselineskip}
\textbf{Palabras clave:} Data Discovery, Business Intelligence, Data Warehouse.