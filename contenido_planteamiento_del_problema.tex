\chapter{Planteamiento del problema}
Actualmente la mayor�a de las instituciones p�blicas e incluso las del sector privado, tienen un bajo nivel de inversi�n en tecnolog�a. Generalmente en empresas que exigen toma de decisi�n, con frecuencia optan por decisiones de negocios no �ptimas debido a que no poseen la suficiente experiencia o suficiente datos procesados del negocio para llegar al correcto an�lisis, o pueden estar usando herramientas incorrectas. Teniendo �nicamente la experiencia como herramienta, puede ser suficiente s�lo en organizaciones peque�as, donde se adquiere conocimiento sin necesidad de alguna herramienta de an�lisis de datos, por ejemplo, sabr�amos cuales son los productos m�s vendidos o m�s rentables en un negocio peque�o, y esto se complica a medida que la organizaci�n crece, cuando la cantidad de sucursales y variedad de productos es amplia, se pierde el control sin ayuda de estas herramientas.\\
La calidad del conocimiento se basa principalmente en la calidad de la informaci�n. Estas informaciones son obtenidas a trav�s de un an�lisis profundo de datos, por consiguiente estos datos tambi�n deben ser de buena calidad. Si una organizaci�n posee la capacidad de obtener lo que desea por medio de los datos, es seguro que su crecimiento ser� positivo, debido a que tomar� mejores decisiones, y estos ayudar�an a su evoluci�n y estabilidad a lo largo del tiempo.\\
Existen ocasiones en que la organizaci�n posee suficientes datos pero no consigue analizarlos o procesarlos por falta de conocimiento del negocio, adem�s de desconocimiento t�cnico de las herramientas.