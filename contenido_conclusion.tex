\pagebreak
\chapter{Conclusiones}
	A lo largo de este trabajo demostramos c�mo es posible dar valor agregado a los datos y en base a estos llegar a tomar decisiones que proporcionen mayores beneficios.\\

	\noindent Teniendo en cuenta los resultados obtenidos se pueden observar las siguientes cuestiones:

\begin{enumerate}
	\item La tasa de crecimiento del consumo es mayor a la tasa de aumento de clientes.
	\item Debido a este aumento en el consumo de energ�a del pa�s, disminuy� la cantidad de energ�a exportada.
	\item Aunque se tuvo una disminuci�n en la energ�a exportada, se obtuvo un crecimiento en el valor facturado. Esto demuestra una mejor�a en el precio de venta de la energ�a al Brasil (Itaip�) o Argentina (Yacyret�).
	\item Seg�n el pron�stico de crecimiento para los pr�ximos a�os, el consumo de energ�a tendr� un crecimiento mayor al de la cantidad de clientes, y la poblaci�n del pa�s.
	\item La tasa de crecimiento de la poblaci�n se mantiene constante durante el tiempo, esto es, la poblaci�n crece a una tasa sostenida. Sin embargo, la tasa de aumento en el consumo de energ�a es considerablemente mayor, y demuestra un aumento abrupto para los pr�ximos a�os, independiente a la tasa de aumento de la poblaci�n y de nuevos clientes de la ANDE.
\end{enumerate}
\noindent
Teniendo en cuenta estas cuestiones, este trabajo puede servir de herramienta para la planificaci�n en la inversi�n en la capacidad de transmisi�n y distribuci�n de electricidad dentro del territorio del pa�s, para los pr�ximos a�os, adem�s de ayudar con indicadores para montar una estrategia de exportaci�n para replantear las condiciones actuales de venta de la energ�a al exterior.\\