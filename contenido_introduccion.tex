\chapter{Introducci�n}
En este trabajo fue elaborada una propuesta de estado del arte del �rea de BI (Business intelligence) y espec�ficamente Data Discovery. El objetivo fue aplicar los conocimientos adquiridos al entorno nacional, en este caso aplicados a una instituci�n del estado (Administraci�n Nacional de Electricidad, ANDE), para el apoyo en la toma de decisiones.
Aplicaremos las t�cnicas de Data Discovery a los datos de dos instituciones del estado, espec�ficamente la \gls{sig:ANDE} y \gls{sig:DGEEC}, donde demostraremos que con datos de calidad podr�amos detectar oportunidades que nos faciliten la toma de decisiones en la instituci�n. Utilizaremos conjuntos de datos de las instituciones mencionadas m�s arriba para este fin.
Este trabajo est� estructurado de la siguiente forma:	
El cap�tulo I presenta el planteamiento del problema, los objetivos generales y espec�ficos, concluyendo con la justificaci�n.
El cap�tulo II presenta el marco te�rico. Se compone de conceptos, componentes, tecnolog�as de BI, y el framework de Gartner.
El cap�tulo III presenta los diferentes paneles indicadores (dashboards) realizados con la herramienta seleccionada para aplicar data discovery sobre los conjuntos de datos existentes. Se explica el cuadrante m�gico de Gartner, el cual apoy� a la selecci�n de la herramienta mencionada. Adem�s se presentan ejemplos de an�lisis de datos y la informaci�n obtenida, como resultado de la combinaci�n del uso de las herramientas y datos.\\