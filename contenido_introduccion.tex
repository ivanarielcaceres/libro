\chapter{Introducci�n}
En este trabajo elaboramos una propuesta de estado del arte del �rea de \gls{sig:BI} y espec�ficamente Data Discovery. El objetivo es aplicar los conocimientos adquiridos al entorno nacional, en este caso aplicados a una instituci�n del estado \gls{sig:ANDE}, para el apoyo en la toma de decisiones.

Aplicamos las t�cnicas de Data Discovery a los datos de dos instituciones del estado, espec�ficamente la \gls{sig:ANDE} y \gls{sig:DGEEC}, donde mostramos que con datos de calidad se pueden detectar oportunidades que nos faciliten la toma de decisiones en la instituci�n. Utilizamos conjuntos de datos de las instituciones mencionadas m�s arriba para este fin.\\

\noindent Este trabajo est� estructurado de la siguiente forma:\\

\begin{itemize}[noitemsep, nolistsep]	
	
\item Planteamiento del problema.

\item Objetivos generales y espec�ficos.

\item Justificaci�n.

\item Marco te�rico. Se compone de conceptos, componentes, tecnolog�as de BI, el framework de Gartner, selecci�n de la herramienta para Data Discovery y se explica el cuadrante m�gico de Gartner, el cual apoy� a la selecci�n de la herramienta mencionada.

\item Resultado: Diferentes paneles indicadores (dashboards) realizados con la herramienta seleccionada para aplicar data discovery sobre los conjuntos de datos existentes. Adem�s se presentan ejemplos de an�lisis de datos y la informaci�n obtenida, como resultado de la combinaci�n del uso de las herramientas y datos.
\end{itemize}