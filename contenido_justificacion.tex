\chapter{Justificaci�n}
{\justifying

Los principales problemas que actualmente la mayor�a de las empresas poseen se describen a continuaci�n:
Falta de un ambiente anal�tico corporativo, que proporcione informaciones e indicadores necesarios para la toma de decisi�n.
Falta de agilidad para elaborar informes con indicadores de tendencias de consumo de energ�a, que apoye a la planificaci�n territorial de expansi�n de la red de transmisi�n el�ctrica.\\

Falta de indicadores de consumo geogr�fico de electricidad, relacionados con indicadores poblacionales, para apoyo en la planificaci�n de inversi�n en el aumento de capacidad de transformadores por regi�n.

El papel principal de BI en las organizaciones es ayudar a que estas puedan comprender sus datos y de esta manera, tomar las decisiones m�s correctas. Este trabajo tiene como objetivo demostrar de manera gr�fica (paneles indicadores) algunas informaciones que podr�an ser �tiles para la organizaci�n. El primer paso para lograr este objetivo es comprender los datos, luego ordenarlos, transformarlos y finalmente visualizarlos.
\par
}