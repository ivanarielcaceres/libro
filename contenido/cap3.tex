
\section{Aplicaci�n de Data Discovery a datos de instituciones del Estado}

\begingroup
\setlength{\intextsep}{-20pt}%
\setlength{\columnsep}{5pt}%
\begin{wrapfigure}{r}{0.25\textwidth}
	\centering
	\includegraphics[width=\linewidth]{figuras/makeDecision}
	\vspace{-10pt}
	\caption{}
	\label{fig:makeDecision}
\end{wrapfigure}
En Paraguay, e inclusive mundialmente, la mayor�a de las empresas no logran comprender 100\% los datos que generan. La consecuencia de no comprender los datos puede conllevar a tomar una decisi�n equivocada y esa decisi�n puede terminar en algo irreversible y de mucho impacto negativo para la organizaci�n a lo largo del tiempo. La informaci�n es considerada como uno de los recursos m�s importantes en una empresa, porque en base a esto se obtiene el conocimiento.

Aplicando la t�cnica de data discovery podremos detectar irregularidades, predecir acontecimientos futuros y preverlos a tiempo. En este trabajo analizaremos los datos de la ANDE y de la DGEEC, realizando cruzamiento entre ambos, con el objetivo de obtener informaci�n de inter�s para la entidad.
\endgroup 

\subsubsection{Datos de la ANDE y de la DGEEC}

Se cuenta con datos de consumo de energ�a el�ctrica, importe facturado, grupo de consumo (residencial, industrial, exportaci�n, etc), por a�o (2000-2014), departamento y distrito. Estos datos fueron solicitados formalmente a la entidad a trav�s de la Facultad de Ciencias y tecnolog�a de la Universidad Cat�lica, la cual tuvimos una respuesta favorable para proceder.

\subsubsection{Dashboard de control / monitoramiento}
	
En esta secci�n se mostrar�n 4 ejemplos de paneles hechos con los datos de ambas instituciones (DGEEC y ANDE).  Una de las t�cnicas de comparaci�n a ser utilizadas en los gr�ficos es ,  ``Medici�n de tasas de crecimientos``, la cual se calcula el porcentaje de crecimiento que hubo por cada a�o. Por ejemplo, si al cerrar el a�o 2014, la cantidad de clientes lleg� a 1.000.000 y en el a�o 2015 aument� 100.000, esto quiere decir que en el a�o 2015, la tasa de crecimiento de los clientes fue del 10\%, es decir, hubo un crecimiento positivo y la cantidad de clientes ha aumentado el respecto al a�o anterior. Suponiendo que el a�o 2016 la ANDE cierra con un total de 1.000.000 clientes,, su crecimiento ser� de 10\% negativo.
La f�rmula empleada es la que sigue, donde ``n`` es el a�o actual y ``n-1`` el a�o anterior, PIB es una variable que indica, en el caso  de nuestra comparaci�n, la cantidad de clientes que posee la ANDE .
	
\textsc{\begin{figure}[H]
	\centering
	\includegraphics[width=0.7\linewidth]{figuras/formula}
	\caption{Form�la para hallar tasa de crecimiento.}
	\label{fig:formula}
\end{figure}}
	
\section{Dashboard - Clientes Facturados vs Crecimiento Poblacional}
Este dashboard se realiz� a fin de cruzar los datos del DGEEC y la ANDE, se utilizan datos hist�ricos de la poblaci�n y datos de clientes, consumo e importe de la ANDE.

\textsc{\begin{figure}[H]
	\centering
	\includegraphics[width=\linewidth]{figuras/ClientesFacturadosVsCrecimientoPoblacional3}
	\caption{Clientes Facturados vs Crecimiento Poblacional}
	\label{fig:ClientesFacturadosVsCrecimientoPoblacional3}
\end{figure}}

El siguiente gr�fico representa el porcentaje del crecimiento anual de los clientes en comparaci�n con el de la poblaci�n. Podemos observar que la l�nea que pertenece a la tasa de crecimiento de la poblaci�n (azul), fue bajando al pasar los a�os. Esto no quiere decir que la poblaci�n fue disminuyendo, sino que cada a�o el porcentaje de aumento es menor.

\textsc{\begin{figure}[H]
	\centering
	\includegraphics[width=\linewidth]{figuras/ClientesFacturadosVsCrecimientoPoblacional4}
	\caption{Clientes Facturados vs Crecimiento Poblacional}
	\label{fig:ClientesFacturadosVsCrecimientoPoblacional4}
\end{figure}}


\textsc{\begin{figure}[H]
	\centering
	\includegraphics[width=\linewidth]{figuras/ClientesFacturadosVsCrecimientoPoblacional}
	\caption{Clientes Facturados vs Crecimiento Poblacional}
	\label{fig:ClientesFacturadosVsCrecimientoPoblacional}
\end{figure}}

La l�nea amarilla representa al porcentaje del crecimiento de los clientes de la ANDE. Como podemos ver, hay      a�os en que el aumento es muy notorio (2004,2006,2008) y hay a�os en que este es m�nimo(2002,2005,2007). Las l�neas discontinuas representan las tendencias de ambos puntos. Por ejemplo, la cantidad de clientes en el a�o 2001 fue de 959.580, la cual aument� el 4.1\% respecto al a�o anterior. 

\textsc{\begin{figure}[H]
	\centering
	\includegraphics[width=\linewidth]{figuras/ClientesFacturadosVsCrecimientoPoblacional2}
	\caption{Clientes Facturados vs Crecimiento Poblacional}
	\label{fig:ClientesFacturadosVsCrecimientoPoblacional2}
\end{figure}}

En el a�o 2002 la cantidad de clientes ascendi� a 964.449 con un aumento de 4.869, que corresponde a un incremento     del 0.5\% respecto  al a�o 2001. Sin embargo, en el a�o 2003 el incremento fue de 1.\%, la cual representa a un aumento de m�s que el doble del a�o anterior.


\textsc{\begin{figure}[H]
	\centering
	\includegraphics[width=\linewidth]{figuras/ClientesFacturadosVsCrecimientoPoblacional5}
	\caption{Clientes Facturados vs Crecimiento Poblacional}
	\label{fig:ClientesFacturadosVsCrecimientoPoblacional5}
\end{figure}}

\textsc{\begin{figure}[H]
	\centering
	\includegraphics[width=\linewidth]{figuras/ClientesFacturadosVsCrecimientoPoblacional6}
	\caption{Clientes Facturados vs Crecimiento Poblacional}
	\label{fig:ClientesFacturadosVsCrecimientoPoblacional6}
\end{figure}}

En el siguiente gr�fico se realiza una proyecci�n o forecasting donde se muestra que probablemente, si la entidad conserva la misma cantidad de clientes o estos crecen m�nimamente, de igual manera el consumo podr�a aumentar o disminuir dr�sticamente. El aumento dr�stico del consumo, podr�a deberse al aumento de productos electr�nicos que consumen mucha m�s energ�a el�ctrica y tambi�n a que el poder adquisitivo de cada ciudadano ha aumentado. Este comportamiento se observa en la zona de c�lculo de proyecci�n (rojo, azul y amarillo suavizado).



\textsc{\begin{figure}[H]
	\centering
	\includegraphics[width=\linewidth]{figuras/EnergiaConsumidaVsImporteFacturado}
	\caption{Energ�a Consumida vs Importe Facturado}
	\label{fig:EnergiaConsumidaVsImporteFacturado}
\end{figure}}

En el tercer y �ltimo gr�fico de este panel, se muestra la porcentaje del crecimiento anual de los importes facturados y consumo de energ�a. Se puede observar que la facturaci�n de la ANDE acompa�a al consumo de energ�a el�ctrica, exceptuando el a�o 2011, en la cual el importe aument� mas de lo que aument� el consumo de energ�a, sin embargo en el a�o 2013 el importe volvi� a aumentar menos que antes.

\textsc{\begin{figure}[H]
	\centering
	\includegraphics[width=\linewidth]{figuras/EnergiaConsumidaVsImporteFacturado2}
	\caption{Energ�a Consumida vs Importe Facturado}
	\label{fig:EnergiaConsumidaVsImporteFacturado2}
\end{figure}}
\subsection{Dashboard - Estad�stica de consumo de electricidad por sector}
En el primer tab, ``Consumo``, tenemos una estad�stica de consumo por grupo de consumidores de energ�a. Al abrir este panel, observamos que la el grupo de consumidores residencial es la que m�s demanda energ�a, hist�ricamente. Otro dato interesante es la exportaci�n de energ�a el�ctrica, que seg�n el gr�fico, cada vez fue disminuyendo m�s. 
A nivel nacional la zona residencial se ubica en primer lugar con un 39,87\%.  Luego viene el sector industrial, que consume el 22.11\% de la energ�a. El sector comercial es due�a del 17,39\%.


\textsc{\begin{figure}[H]
	\centering
	\includegraphics[width=\linewidth]{figuras/EstadisticaDeConsumoDeElectricidadPorSector1990-2014}
	\caption{Estad�stica de consumo de electricidad por sector(1990-2014)}
	\label{fig:EstadisticaDeConsumoDeElectricidadPorSector1990-2014}
\end{figure}}

En el segundo tab, titulado ``Facturaci�n``, tenemos datos de facturaciones por grupo de consumidores.  Al igual que el gr�fico anterior observamos que la zona residencial es a la que m�s facturas se emiten.


\textsc{\begin{figure}[H]
	\centering
	\includegraphics[width=\linewidth]{figuras/ImporteFacturadoPorAnoYSector1990-2014}
	\caption{Importe facturado por a�o y sector(1990-2014)}
	\label{fig:ImporteFacturadoPorAnoYSector1990-2014}
\end{figure}}
\subsubsection{Panel comparativo de Tasa de crecimiento y el Consumo de energ�a}

En la figura ~\ref{fig:TasaDeCrecimientoVsConsumoDeEnergia} se presenta un panel comparativo entre la tasa de crecimiento de clientes y consumo de energ�a. Tambi�n se cuenta con un mapa para filtrar por regi�n del pa�s. 

\textsc{\begin{figure}[H]
	\centering
	\includegraphics[width=\linewidth]{figuras/TasaDeCrecimientoVsConsumoDeEnergia}
	\caption{Tasa de crecimiento vs Consumo de energ�a}
	\label{fig:TasaDeCrecimientoVsConsumoDeEnergia}
\end{figure}}

En el gr�fico situado a la derecha del mapa (~\ref{fig:TasaDeCrecimientoVsConsumoDeEnergia2}), se tiene el consumo y crecimiento de clientes. Para un mejor analisis fue necesario suavizar los datos calculando lineas de tendencia, debido a una inestabilidad de los datos de consumo. As� se puede apreciar, que existe una tendencia de crecimiento sostenido durante el tiempo de clientes. Sin embargo, la tendencia que el consumo crezca es mayor al de clientes.

\textsc{\begin{figure}[H]
	\centering
	\includegraphics[width=\linewidth]{figuras/TasaDeCrecimientoVsConsumoDeEnergia2}
	\caption{Tasa de crecimiento vs Consumo de energ�a, filtrado por el departamento Alto Paran�}
	\label{fig:TasaDeCrecimientoVsConsumoDeEnergia2}
\end{figure}}

Debajo se presenta un cuadro con la l�nea de cada valor (Figura ~\ref{fig:ProyeccionDeClientesYConsumosParaLosProximos5Anos}), de crecimiento y consumo. Con el uso de un recurso disponible que cuenta la herramienta escogida Tableau, es posible realizar an�lisis predictivo  (forecasting) del crecimiento y consumo. Tableau utiliza un algoritmo llamado Suavizado Exponencial, muy conocido en el �rea de Matem�ticas Estad�sticas. En este gr�fico se puede notar que existe una mayor probabilidad que en los pr�ximos a�os aumente considerablemente el consumo, superando su media. Sin embargo, se nota que el ritmo de crecimiento de clientes es sostenible, y no tiene una alta probabilidad de sufrir un aumento abrupto.

\textsc{\begin{figure}[H]
		\centering
		\includegraphics[width=\linewidth]{figuras/ProyeccionDeClientesYConsumosParaLosProximos5Anos}
		\caption{Proyecci�n de clientes y consumos para los pr�ximos 5 a�os}
		\label{fig:ProyeccionDeClientesYConsumosParaLosProximos5Anos}
	\end{figure}}
	
\subsubsection{Panel de Tasa de crecimiento poblacional y Consumo de energ�a anual}

\textsc{\begin{figure}[H]
	\centering
	\includegraphics[width=\linewidth]{figuras/TasaDeCrecimientoPoblacionalYConsumoDeEnergiaAnoTrasAno}
	\caption{Panel de Tasa de crecimiento poblacional y Consumo de energ�a anual}
	\label{fig:TasaDeCrecimientoPoblacionalYConsumoDeEnergiaAnoTrasAno}
\end{figure}}

En la figura~\ref{fig:TasaDeCrecimientoPoblacionalYConsumoDeEnergiaAnoTrasAnoMapa} se presenta el panel comparativo de datos de crecimiento poblacional de la DGEEC y de consumo de la ANDE. En el mapa, cuando el color es m�s oscuro el consumo de energ�a es mayor, y si el color es m�s claro, el consumo es menor. Se puede notar que los departamentos Central y Alto Paran� son los que tiene mayor consumo de energ�a. 
Este tipo de gr�fico es muy �til cuando se desea analizar informaci�n de forma general y georeferenciada. Al ubicar el mouse sobre cualquier departamento, se presenta una ventana emergente indicando el valor de consumo del departamento seleccionado. Al seleccionar un departamento del mapa, los dem�s gr�ficos tambi�n se actualizan filtrando el departamento seleccionando. 

\textsc{\begin{figure}[H]
	\centering
	\includegraphics[width=\linewidth]{figuras/TasaDeCrecimientoPoblacionalYConsumoDeEnergiaAnoTrasAnoMapa}
	\caption{Consumo por departamento}
	\label{fig:TasaDeCrecimientoPoblacionalYConsumoDeEnergiaAnoTrasAnoMapa}
\end{figure}}
\noindent
 Los gr�ficos situados a la derecha y debajo del mapa presentan la tasa de crecimiento poblacional y de consumo de energ�a. Es importante notar que la tasa de crecimiento de la poblaci�n es casi constante, esto es, no tiene una gran variaci�n en el tiempo, ni una tendencia a alejarse de la media. Sin embargo, la tasa de crecimiento de consumo tiene una l�nea de tendencia a crecer durante el tiempo. As� puede apreciarse que, el consumo de energ�a no tiene una relaci�n de proporcionalidad con respecto al crecimiento de la poblaci�n.

\textsc{\begin{figure}[H]
	\centering
	\includegraphics[width=\linewidth]{figuras/ProyeccionDeCrecimientoPoblacionalYConsumoDeEnergia}
	\caption{Proyecci�n de crecimiento poblacional y consumo ee energ�a}
	\label{fig:ProyeccionDeCrecimientoPoblacionalYConsumoDeEnergia}
\end{figure}}
\noindent
En este gr�fico,  se muestra la misma informaci�n que el gr�fico anterior pero con diferente perspectiva, en este caso se calcula el porcentaje de crecimiento anual tanto de la poblaci�n, as� como del consumo.


\textsc{\begin{figure}[H]
	\centering
	\includegraphics[width=\linewidth]{figuras/ProyeccionDeClientesYConsumosParaLosProximos5Anos2}
	\caption{Tasa de crecimiento poblacional y consumo de energ�a a�os tras a�os}
	\label{fig:ProyeccionDeClientesYConsumosParaLosProximos5Anos2}
\end{figure}}
