\pagebreak
\chapter{CAPITULO 4}
\section{Marco Metodol�gico}
\subsection{Alcance}

\subsection{Enfoque}	
El enfoque que utilizamos es el cuantitativo, que por lo com�n, utiliza la recolecci�n y el an�lisis de datos para contestar
preguntas de investigaci�n y probar hip�tesis establecidas previamente, y conf�a en la medici�n num�rica, el conteo, y en el uso de la estad�stica para intentar establecer con exactitud  patrones en una poblaci�n. (por ejemplo un censo es un enfoque cuantitativo del estudio demogr�fico de la poblaci�n de un pa�s).\cite{gomez2006introduccion} 
\subsection{T�cnica e Instrumentos de recolecci�n de datos}
La t�cnica aplicada en este trabajo en la recolecci�n de datos fue la investigaci�n de documentos cient�ficos procedentes de publicaciones de empresas pioneras en Data Discovery y de expertos en el �rea.
