\chapter{Introducci�n}
Esta tesis se basa en un trabajo de investigaci�n sobre el estado del arte de lo que denominamos BI (Business intelligence) y data discovery. El objetivo es aplicar los conocimientos adquiridos a nuestro entorno, en este caso aplicados a una instituci�n del estado (la ANDE), para el apoyo a la hora de tomar decisiones.

Este trabajo presenta los siguientes cap�tulos:
El cap�tulo I presenta el planteamiento del problema, los objetivos generales y espec�ficos, se hablar� sobre los conceptos, componentes, tecnolog�as de BI, el framework de Gartner, la cual fue preparado por una de las consultoras l�deres en BI.
El cap�tulo II presenta los diferentes dashboards realizados con la herramienta seleccionada para aplicar data discovery sobre los conjuntos de datos existentes. Se explicar� el cuadrante m�gico de Gartner, el cual ayud� a la selecci�n de la herramienta mencionada. Se demostrar�n ejemplos de an�lisis de datos y la informaci�n capaz de obtener si se hace un buen uso de la herramientas y los datos subyacentes.
\section{Planteamiento del problema}
Actualmente la mayor parte de las instituciones estatales e incluso del sector privado no invierten lo suficiente en tecnolog�a. La mayor�a de las empresas contratan personas que tomen decisiones correctas en la situaci�n que se presente. Estas personas no pueden ejecutar ninguna acci�n correcta si es que no poseen la experiencia en el negocio o  los datos correctos para analizar. La experiencia resulta en organizaciones peque�as, donde con un tiempo se adquiere conocimiento sin necesidad de un apoyo digital, por ejemplo, los productos m�s vendidos o m�s rentables de un negocio. Esto se complica a medida que la organizaci�n crece. Cuando la cantidad de sucursales y variedad de productos es amplio ya se pierde el control de manejarlo sin ayuda de la tecnolog�a. 

La calidad del conocimiento se basa principalmente en que  la informaci�n sea de calidad. Estas informaciones son obtenidas a trav�s de un an�lisis profundo de los datos, por consiguiente estos datos tambi�n deben ser de buena calidad. Si una organizaci�n posee la capacidad de obtener lo que desea por medio de los datos, es seguro que su crecimiento ser� positivo, debido a que tomar� mejores decisiones, y estos ayudar�an a su evoluci�n a lo largo del tiempo.

Existen ocasiones en que la organizaci�n posee los datos suficientes pero no consigue analizarlos o procesarlos por falta conocimiento ya sea t�cnico o de negocio.