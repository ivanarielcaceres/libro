\chapter{CAPITULO 1}
\section{Introducci�n}
En este trabajo fue elaborada una propuesta de estado del arte del �rea de BI (Business intelligence) y espec�ficamente Data Discovery. El objetivo fue aplicar los conocimientos adquiridos al entorno nacional, en este caso aplicados a una instituci�n del estado (Administraci�n Nacional de Electricidad, ANDE), para el apoyo en la toma de decisiones.
Este trabajo est� estructurado de la siguiente forma:	
El cap�tulo I presenta el planteamiento del problema, los objetivos generales y espec�ficos, concluyendo con la justificaci�n.
El cap�tulo II presenta el marco te�rico. Se compone de conceptos, componentes, tecnolog�as de BI, y el framework de Gartner.
El cap�tulo III presenta los diferentes paneles indicadores (dashboards) realizados con la herramienta seleccionada para aplicar data discovery sobre los conjuntos de datos existentes. Se explica el cuadrante m�gico de Gartner, el cual apoy� a la selecci�n de la herramienta mencionada. Adem�s se presentan ejemplos de an�lisis de datos y la informaci�n obtenida, como resultado de la combinaci�n del uso de las herramientas y datos.
\section{Planteamiento del problema}
Actualmente la mayor�a de las instituciones p�blicas e incluso las del sector privado, tienen un bajo nivel de inversi�n en tecnolog�a. Generalmente en empresas que exigen toma de decisi�n, con frecuencia optan por decisiones de negocios no �ptimas debido a que no poseen la suficiente experiencia o suficiente datos procesados del negocio para llegar al correcto an�lisis, o pueden estar usando herramientas incorrectas. Teniendo �nicamente la experiencia como herramienta, puede ser suficiente s�lo en organizaciones peque�as, donde se adquiere conocimiento sin necesidad de alguna herramienta de an�lisis de datos, por ejemplo, sabriamos cuales son los productos m�s vendidos o m�s rentables en un negocio peque�o, y esto se complica a medida que la organizaci�n crece, cuando la cantidad de sucursales y variedad de productos es amplia, se pierde el control sin ayuda de estas herramientas. 

La calidad del conocimiento se basa principalmente en la calidad de la informaci�n. Estas informaciones son obtenidas a trav�s de un an�lisis profundo de datos, por consiguiente estos datos tambi�n deben ser de buena calidad. Si una organizaci�n posee la capacidad de obtener lo que desea por medio de los datos, es seguro que su crecimiento ser� positivo, debido a que tomar� mejores decisiones, y estos ayudar�an a su evoluci�n y estabilidad a lo largo del tiempo.

Existen ocasiones en que la organizaci�n posee suficientes datos pero no consigue analizarlos o procesarlos por falta de conocimiento del negocio, adem�s de desconocimiento t�cnico de las herramientas.