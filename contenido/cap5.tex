\pagebreak
\chapter{CAPITULO 5}
\section{Conclusiones y Trabajos futuros}
Este trabajo fue dividido en las siguientes partes:

- Capítulo 1: en este capítulo fue presentada una introducción a los objetivos de este trabajo.

- Capítulo 2: Marco Teórico: esta sección desarrolla el estado del arte en el área de BI y principalmente el de Data Discovery, técnica aplicada en este trabajo.

- Capítulo 3: En esta sección se presenta la evaluación técnica realizada para la selección de la herramienta Tableau. Además la aplicación de técnicas de Data Discovery a los datos de la ANDE y la DGEEC. También se presentan los productos construidos en este trabajo, para el análisis de las informaciones, y las proyecciones realizadas para los próximos años.

- Capítulo 4: ........

Para el desarrollo de este trabajo se obtubieron datos de la ANDE y la DGEEC. Con estos datos fue posible aplicar técnicas de Data Discovery realizar un analisis de las informaciones, cruzarlos, georeferenciarlos, encontrar líneas de tendencias y pronósticos de crecimiento a futuro tanto del consumo de energía, de clientes de la ANDE y de la población del país. En cada caso se presenta un análisis que demuestra con gráficos intuitivos que en ciertas ocasiones no existe una relacion proporcional entra algunas dimensiones. Sin embargo, teniendo en cuenta los resultados obtenidos se pueden observar las siguientes cuestiones:

- La tasa de crecimiento del consumo es mayor a la tasa de aumento de clientes 
- Debido a este aumento en el consumo de energía del país, disminuyó la cantidad de energía exportada.
- Aunque se tuvo una disminución en la energía exportada, se obtuvo un crecimiento en el valor facturado. Esto demuestra una mejoría en el precio de venta de la energía al Brasil (Itaipu) o Argentina (Yacyreta).
- Según el pronóstico de crecimiento para los próximos años, el consumo de energía tendrá un crecimiento mayor al de la cantidad de clientes, y la población del país.
- La tasa de crecimiento de la población se mantiene constante durante el tiempo, esto es, la población crece a una tasa sostenida. Sin embargo, la tasa de aumento en el consumo de energía es considerablemente mayor, y demuestra un aumento abrupto para los próximos años, independiente a la tasa de aumento de la población y de nuevos clientes de la ANDE.


Teniendo en cuenta estas cuestiones, este trabajo puede servir de herramienta para la Planificación en la Inversión en la capacidad de transmisión y distribución de electricidad dentro del territorio del país, para los próximos años, además de ayudar con indicadores para montar una estrategia de exportación para replantear las condiciones actuales de venta de la energía al exterior.