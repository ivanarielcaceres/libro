\pagebreak
\chapter{Marco Metodol�gico}
\section{Alcance}
Aplicamos las t�cnicas de Data Discovery a los datos de dos instituciones del estado, espec�ficamente la \gls{sig:ANDE} y \gls{sig:DGEEC}, donde mostramos que con datos de calidad se pueden detectar oportunidades que nos facilitan la toma de decisiones en la instituci�n. Utilizamos conjuntos de datos de las instituciones mencionadas m�s arriba para este fin.
\section{Enfoque}	
El enfoque que utilizamos es el cuantitativo.\\
\noindent (...) que por lo com�n, usa la recolecci�n y el an�lisis de datos para contestar preguntas de investigaci�n y probar hip�tesis establecidas previamente, y conf�a en la medici�n num�rica, el conteo, y en el uso de la estad�stica para intentar establecer con exactitud  patrones en una poblaci�n. \cite{monge2010estudio}
\section{T�cnica e Instrumentos de recolecci�n de datos}
La t�cnica aplicada en este trabajo en la recolecci�n de datos fue la investigaci�n de documentos cient�ficos procedentes de publicaciones de empresas pioneras en Data Discovery y de expertos en el �rea.
