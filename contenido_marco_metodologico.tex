\vfill
\pagebreak
\chapter{Marco Metodol�gico}
\section{Alcance}
Aplicamos las t�cnicas de Data Discovery a los datos de dos instituciones del estado, espec�ficamente la \gls{sig:ANDE} y \gls{sig:DGEEC}, donde mostramos que con datos de calidad se pueden detectar oportunidades que nos facilitan la toma de decisiones en la instituci�n. Utilizamos conjuntos de datos de las instituciones mencionadas m�s arriba para este fin.

De acuerdo al tema planteado, el alcance de la investigaci�n es el descriptivo, definida por \cite{hernandez2010pilar}.
\section{Dise�o de la investigaci�n}	

 Esta investigaci�n es de  dise�o no experimental, transversal pues no se trata de manipular las variables que intervienen en la investigaci�n, sino estudiar el estado natural en que se encuentra. Tiene el prop�sito la recolecci�n de datos en un �nico momento y despu�s analizar el contexto\cite{hernandez2010pilar}.
\section{T�cnica e Instrumentos de recolecci�n de datos}
La t�cnica aplicada en este trabajo en la recolecci�n de datos fue la investigaci�n de documentos procedentes de publicaciones de empresas pioneras en Data Discovery y de expertos en el �rea.
\section{Poblaci�n}
El universo de datos analizado fue el consumo de energ�a el�ctrica, facturaciones y datos relacionados a la poblaci�n del Paraguay.
Los datos pertenecientes a la ANDE fueron solicitados formalmente por medio de la Facultad de Ciencias y Tecnolog�a de la Universidad Cat�lica, a la cual tuvimos una respuesta favorable.

\section{Muestra}
Las muestras se pueden clasificar por departamento y distrito y tambi�n en residencial, industrial, exportaci�n, comercial, gubernamental y otros. Estas muestras son a partir del a�o 2000 hasta el 2014.